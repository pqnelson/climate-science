% 17
\section{Wave Equation}

\begin{node}[Derivation]\label{pde:wave-0000}%
There are multiple derivations of the wave equation in 1 spatial
dimension.

\begin{node}[Springs]\label{pde:wave-0001}%
Consider an infinite system of point masses (of mass $m$) connected by
identical massless springs (with spring constant $k$) such that they
form a line. We would have (writing $x_{i}(t)$ for the position of the
$i^{\text{th}}$ point-mass),
\begin{center}
  \includegraphics{img/wave-equation-0.mps}
\end{center}
Then using Hooke's law, the force acting on the $i^{\text{th}}$
point-mass $m$ whose position is $x_{i}(t)$ will be,
\begin{subequations}
\begin{equation}
\begin{split}
F_{\text{Hooke}} &= F_{i+1}-F_{i-1}\\
&=k\bigl(x_{i+1}(t)-x_{i}(t)\bigr)-k\bigl(x_{i}(t)-x_{i-1}(t)\bigr).
\end{split}
\end{equation}
Then Newton's laws of motion gives us for the point-mass with
position $x_{i}$,
\begin{equation}
\frac{\D^{2}}{\D t^{2}}x_{i}(t) = \frac{k}{m}\bigl(x_{i+1}(t)-2x_{i}(t)+x_{i-1}(t)\bigr).
\end{equation}
If our system of $N$ point masses spaced evenly over the length
$L=N\Delta x$ of total mass $M=Nm$, and the spring constant of the array
$K=k/N$, we then have:
\begin{equation}
\frac{\D^{2}}{\D t^{2}}x_{i}(t) = 
\frac{KL^{2}}{M}\left(\frac{x_{i+1}(t)-2x_{i}(t)+x_{i-1}(t)}{(\Delta x)^{2}}\right).
\end{equation}
Note that we could equally as well work with the \emph{displacement from equilibrium}
$u_{i}(x_{i,\text{eq}},t)$ instead of the current position of the $i^{\text{th}}$
point-mass $x_{i}(t)$ by observing $u_{i}(x_{i,\text{eq}},t)=x_{i}(t)-x_{i,\text{eq}}$.
Since the point-masses being in equilibrium would not experience Hooke's
law, their contribution would subtract out from Hooke's force law, and
we'd end up with
\begin{equation}
\partial_{t}^{2}u_{i}(x_{i,\text{eq}},t) = 
\frac{KL^{2}}{M}\left(\frac{u_{i+1}(x_{i+1,\text{eq}},t)-2u_{i}(x_{i,\text{eq}},t)+u_{i-1}(x_{i-1,\text{eq}},t)}{(\Delta x)^{2}}\right).
\end{equation}
Taking the continuum limit $N\to\infty$, $\Delta x\to0$ transforms
$x_{i}(t)\to u(x,t)$ since each point become the equilibrium position
for some point-mass, and the equations of motion become:
\begin{equation}
\boxed{\partial_{t}^{2}u(x,t)=\frac{KL^{2}}{M}\partial_{x}^{2}u(x,t).}
\end{equation}
\end{subequations}
This is the wave equation, and $c=[KL^{2}/M]^{1/2}$ is the speed of propagation.
\end{node}

\begin{node}[Problematic derivations]\label{pde:wave-0002}%
There are a number of issues with the standard derivation of the wave
equation for a vibrating string. Yong~\cite{yong2006strings} gives a
derivation starting from physically correct first principles, and
reviews the literature behind it.
\end{node}

\begin{node}[Waves in general]\label{pde:wave-0003}%
Broadly speaking, ``wave motion'' describes the form of a
\emph{solution} rather than a generic \emph{problem}. There are two
types of waves: \emph{hyperbolic} waves (which is the kind described by
the equation $\partial_{t}^{2}u-\laplacian u=f$) and \emph{dispersive}
waves which are described by solutions of the form $u=a\cos(\omega t-\vec{k}\cdot\vec{x})$
where $\omega$ and $\vec{k}$ may be functions of space and time.
For more on wave motion in general, see Whitham~\cite{whitham1974waves}.
\end{node}
\end{node}

\begin{node}[In one spatial-dimension]\label{pde:wave-0007}%
We will study the wave equation in one spatial-dimension, since many of
its properties hold in $n>1$ spatial dimensions, and it avoids needless
complexities.

\begin{node}[General Solution]\label{pde:wave-0004}%
We see that we can, for constant speed of propagation $c$, write
\begin{equation}
(\partial_{t}^{2}-c^{2}\partial_{x}^{2})u=(\partial_{t}-c\partial_{x})
(\partial_{t}+c\partial_{x})u=0.
\end{equation}
Then the general solution looks like
\begin{equation}
u(x,t) = f(x-ct) + g(x+ct)
\end{equation}
for any $f,g\in C^{1}(\RR)$ since $(\partial_{t}+c\partial_{x})f(x-ct)=0$ and
$(\partial_{t}-c\partial_{x})g(x+ct)=0$ identically.

Observe that $f(x+ct)$ is moving to the left at speed $c$, and $g(x-ct)$
is moving to the right with speed $c$. This can be understood by looking
at the characteristic curves $x\pm ct=C_{0}$ where $C_{0}$ is constant,
then $ct=C_{0}\mp x$, so as we move forward in time we must move to the
left or right according to the sign of $\mp x$.
\end{node} % general solution

\begin{node}[Initial value problem]\label{pde:wave-0005}%
If we consider the initial value problem for the one-dimensional wave,
we need to give the initial condition
\begin{equation}
u(x,0)=\phi(x),\quad\mbox{and}\quad\partial_{t}u(x,0)=\psi(x),
\end{equation}
where it is understood $\partial_{t}u(x,0)=\partial_{t}u(x,t)|_{t=0}$.
In this situation, there exists exactly one solution to the wave
equation satisfying the initial condition.

We can find it by first setting $t=0$, which gives us
\begin{subequations}
\begin{equation}
\phi(x)=f(x)+g(x).
\end{equation}
Then we can differentiate the general solution with respect to $t$, and
at $t=0$ find
\begin{equation}
\psi(x)=cf'(x)-cg'(x).
\end{equation}
This is a system of two equations in two unknowns. We can differentiate
the first equation
\begin{equation}
\phi'(x)=f'(x)+g'(x),
\end{equation}
and then simple algebra gives us
\begin{equation}
f'(x)=\frac{\phi'(x)+c^{-1}\psi(x)}{2},
\end{equation}
and
\begin{equation}
g'(x)=\frac{\phi'(x)-c^{-1}\psi(x)}{2}.
\end{equation}
Integration gives us
\begin{equation}
f(s)=\frac{1}{2}\phi(s)+\frac{1}{2c}\int^{s}_{0}\psi(v)\,\D v + C_{1}
\end{equation}
and
\begin{equation}
g(s)=\frac{1}{2}\phi(s)-\frac{1}{2c}\int^{s}_{0}\psi(v)\,\D v + C_{2}
\end{equation}
where $C_{1}$, $C_{2}$ are constants of integration which, since
$\phi(x+ct)=f(x+ct)+g(x-ct)$, must sum to zero $C_{1}+C_{2}=0$.
\end{subequations}
Therefore we obtain the solution to the initial value problem,
\begin{equation}\label{eq:pde:wave:one-dim:initial-value-problem:soln}
\boxed{u(x,t)=\frac{\phi(x+ct)+\phi(x-ct)}{2}+\frac{1}{2c}\int^{x+ct}_{x-ct}\psi(s)\,\D s.}
\end{equation}
We see when $t=0$ this reduces to
$u(x,0)=\phi(x)+\int^{x}_{x}\psi(s)\,\D s=\phi(x)$ as desired, and
similarly the time derivative of $u(x,t)$ evaluated at $t=0$ is equal to
$\psi(x)$.

\begin{node}[Example: piano]\label{pde:wave-000F}%
When $\phi=0$ and $\psi\neq0$, this models what happens when you press a
piano key: a ``hammer'' hits a chord with a certain velocity when the
chord is at rest.
\end{node}

\begin{node}\label{pde:wave-000E}%
If we had a forcing term to our wave equation,
\begin{equation}
\partial_{t}^{2}u=c^{2}\partial_{x}^{2}u + F(x,t),
\end{equation}
then the solution to the initial value problem with $u(x,0)=\phi(x)$ and
$\partial_{t}u(x,0)=\psi(x)$ is
\begin{equation}\label{eq:pde:wave:1-d:ivp:forcing}
\begin{split}
u(x,t)&=\frac{\phi(x+ct)-\phi(x-ct)}{2}+\frac{1}{2c}\int^{x+ct}_{x-ct}\psi(y)\,\D y\\
&\qquad+\frac{1}{2c}\int^{t}_{0}\int^{x+c(t-s)}_{x-c(t-s)}F(y,s)\,\D y\, \D s.
\end{split}
\end{equation}
\begin{proof}[Proof sketch]
Taking advantage of linearity, we can consider the initial value problem
\begin{subequations}
\begin{equation}
\partial_{t}^{2}v=c^{2}\partial_{x}^{2}v+F(x,t)
\end{equation}
where $v(x,0)=0$ and $\partial_{t}v(x,0)=0$, then add this to the
unforced solution to the initial value problem
$\partial_{t}^{2}u_{\text{free}}=c^{2}\partial_{x}^{2}u_{\text{free}}$ with $u_{\text{free}}(x,0)=\phi(x)$ and
$\partial_{t}u_{\text{free}}(x,0)=\psi(x)$. That is,
$u(x,t)=u_{\text{free}}(x,t)+v(x,t)$ by linearity, and we know
$u_{\text{free}}(x,t)$, so we just need to find $v(x,t)$.

Changing coordinates to $\xi=x-ct$ and $\eta=x+ct$, we have
$\partial_{t}^{2}-c^{2}\partial_{x}^{2}=-4c^{2}\partial_{\xi}\partial_{\eta}$.
This gives us
\begin{equation}
\partial_{\xi}\partial_{\eta}=\frac{-1}{4c^{2}}F\left(\frac{\eta+\xi}{2},\frac{\eta-\xi}{2c}\right).
\end{equation}
Integrate both sides with respect to $\eta$ on the interval $[\xi,\eta]$
\begin{equation}
\partial_{\xi}v(\xi,\eta)-\partial_{\xi}v(\xi,\xi)
=\frac{-1}{4c^{2}}\int^{\eta}_{\xi}F\left(\frac{z+\xi}{2},\frac{z-\xi}{2c}\right)
\D z.
\end{equation}
But $\partial_{\xi}v(\xi,\xi)$ is just a linear combination of the
initial conditions, which means it vanishes:
\begin{equation}
\partial_{\xi}v(\xi,\xi)=\frac{1}{2c}\partial_{t}v(\xi,0)+\frac{1}{2}\partial_{x}v(\xi,0)=0+0=0.
\end{equation}
Therefore we need to solve:
\begin{equation}
\partial_{\xi}v(\xi,\eta)
=\frac{-1}{4c^{2}}\int^{\eta}_{\xi}F\left(\frac{z+\xi}{2},\frac{z-\xi}{2c}\right)
\D z.
\end{equation}
Integrating for $\xi$ over the interval $[\xi,\eta]$ gives us
\begin{equation}
\int^{\eta}_{\xi}\partial_{w}v(w,\eta)\,\D w
=\frac{-1}{4c^{2}}\int^{\eta}_{\xi}\int^{\eta}_{\xi}F\left(\frac{z+w}{2},\frac{z-w}{2c}\right)
\D z\,\D w
\end{equation}
The left-hand side of this reduces to (since $v(\eta,\eta)=0$)
\begin{equation}
-v(\xi,\eta)
=\frac{-1}{4c^{2}}\int^{\eta}_{\xi}\int^{\eta}_{\xi}F\left(\frac{z+w}{2},\frac{z-w}{2c}\right)
\D z\,\D w.
\end{equation}
Now we do a change of coordinates $w=y-cs$ and $z=y+cs$, and the domain
of integration is precisely $x-ct\leq y-cs\leq y+cs\leq x+ct$, the
Jacobian determinant is $2c$, and the result follows by returning to our
original coordinates.
\end{subequations}
\end{proof}
\end{node} % forcing term

\begin{example}[{Resonance~\cite[2.19]{olver2016intro}}]
Consider the initial value problem
\begin{equation*}
\partial_{t}^{2}u=c^{2}\partial_{x}^{2}+\frac{c^{2}}{\ell}\sin(\omega t)\sin(xk),
\end{equation*}
with $u(x,0)=0$ and $\partial_{t}u(x,0)=0$, $k$ is some constant of
dimensions ``per unit length'' (for dimensionality reasons), and
$\omega>0$ is a fixed frequency. The general solution is then
\begin{calculation}
  u(x,t)
\step{using Eq~\eqref{eq:pde:wave:1-d:ivp:forcing}}
  \frac{1}{2c}\int^{t}_{0}\int^{x+c(t-s)}_{x-c(t-s)}\sin(\omega s)\sin(k y)\,\D y\,\D s
\step{since $\int\sin(y)\D y=-\cos(y)$}
  \frac{1}{2c}\frac{1}{k}\frac{-c}{\omega}\int^{t}_{0}\sin(\omega s)\bigl(\cos(c^{-1}\omega x-\omega t+\omega s)-\cos(c^{-1}\omega x+\omega t-\omega s)\bigr)\,\D s
\end{calculation}
Then depending on if $k\omega=c$ or not, we have two different possible
solutions
\begin{equation}
u(x,t)
= \begin{cases}\displaystyle\frac{c}{\omega^{2}}(\sin(t\omega)-t\omega\cos(t\omega))\sin(\omega x/c)
&\mbox{if } k\omega=c\\[2ex]
\displaystyle\frac{2}{k}\frac{ck\sin(t\omega)-\omega\sin(ckt)}{c^{2}k^{2}-\omega^{2}}\sin(kx)
&\mbox{if } 0<k\omega\neq c
\end{cases}
\end{equation}
Observe that if $k\omega=c$, then $u(x,t)\sim -t\omega\cos(t\omega)\sin(\omega x/c)$
plus bounded terms, which means that $u(x,t)$ grows without bound over
time. In this case, $\omega$ is a \define{Resonant Frequency}.

However, if $\omega k\neq c$, then $u(x,t)$ \emph{is bounded}. In this
case, the solution is periodic if $\omega k/c$ is rational, and
\emph{quasiperiodic} if $\omega k/c$ is irrational (meaning it will
never repeat itself).
\end{example}
\end{node} % initial value problem




\begin{node}[Energy]\label{pde:wave-0009}%
The energy
\begin{equation}
E = \int\frac{\rho}{2}(\partial_{t}u)^{2}+\frac{c^{2}\rho^{2}}{2}(\partial_{x}u)^{2}\,\D
x
\end{equation}
is a constant with respect to time.

\begin{proof}
Let $\rho$ be the mass density for the wave, and suppose it is a constant.
We have the kinetic energy be obtained by integrating
$\frac{1}{2}\rho(\partial_{t}u)^{2}$ at each point in the wave
\begin{subequations}
\begin{equation}
K = \int\frac{\rho}{2}(\partial_{t}u)^{2}\,\D x.
\end{equation}
We compute its time derivative,
\begin{calculation}
\frac{\D K}{\D t}
\step{unfolding the definition of $K$}
\int\rho(\partial_{t}u)(\partial_{t}^{2}u)\,\D x
\step{using $\partial_{t}^{2}u=c^{2}\partial_{x}^{2}u$}
\int\rho(\partial_{t}u)(c^{2}\partial_{x}^{2}u)\,\D x
\step{integrate by parts, $f=\partial_{t}u$, $g=\partial_{x}u$}
\left.\rho c^{2}(\partial_{t}u)(\partial_{x}u)\right|^{x=+\infty}_{x=-\infty}
-\int\rho(\partial_{x}\partial_{t}u)(c^{2}\partial_{x}u)\,\D x
\step{boundary terms vanish}
-\int\rho(\partial_{x}\partial_{t}u)(c^{2}\partial_{x}u)\,\D x
\step{using product rule}
-\int\partial_{t}\left(\frac{1}{2}c^{2}\rho(\partial_{x}u)^{2}\right)\D x
\step{pulling out the time derivative}
-\frac{\D}{\D t}\int\left(\frac{1}{2}c^{2}\rho(\partial_{x}u)^{2}\right)\D x.
\end{calculation}
If we identify the potential energy $V$ as
\begin{equation}\label{eq:pde:wave:energy:identify-potential-energy}
V := \int\left(\frac{1}{2}c^{2}\rho(\partial_{x}u)^{2}\right)\D x,
\end{equation}
then $E=K+V$ has its time derivative equal to zero identically, which
implies the result.
\end{subequations}
\end{proof}

\begin{node}[Potential energy]\label{pde:wave-000C}%
Note that the potential energy in
Eq~\eqref{eq:pde:wave:energy:identify-potential-energy} seems \textit{ad hoc},
as if it were chosen for convenience to prove the claim. But if we
examine the derivation~\zref{pde:wave-0001} of the wave equation using a
system of springs, and if we examine the continuum limit of the
Lagrangian mechanics for the system, then we will obtain Eq~\eqref{eq:pde:wave:energy:identify-potential-energy}
as the potential energy for the system. So it's not completely unjustified.

But finding nontrivial constants with respect to time is valuable, so
even if we had some \textit{ad hoc} manipulations (which we didn't),
then the result is worth keeping.
\end{node}

\begin{node}[In higher dimensions]\label{pde:wave-000D}%
The energy in higher dimensions may be generalized for scalar wave equations to
\begin{equation*}
E = \frac{1}{2}\int((\partial_{t}u)^{2}+c^{2}|\grad u|^{2})\,\D^{n}x.
\end{equation*}
The exact same arguement as in 1-spatial dimension works to show that it
is a constant with respect to time (i.e., energy is conserved).
\end{node}
\end{node} % Energy

\end{node} % in one spatial dimensions


\begin{node}[In three spatial-dimensions]\label{pde:wave-0008}%
There are two ways to even pose the generalization to three
spatial-dimensions: we could work with a scalar wave $u(\vec{x},t)$ or a
vector wave $\vec{u}(\vec{x},t)$.

\begin{node}[Scalar wave]\label{pde:wave-000A}%
The trick is to realize the wave now has a velocity of propagation
vector $\vec{c}$. Assuming that $\vec{c}$ is a constant vector, we
factorize the wave equation similarly as:
\begin{subequations}
\begin{equation}
(\partial_{t}^{2}-\vec{c}\cdot\vec{c}\laplacian)u
=(\partial_{t}-\vec{c}\cdot\grad)(\partial_{t}+\vec{c}\cdot\grad)u=0.
\end{equation}
The initial value problem
\begin{equation}
u(\vec{x},0)=\phi(\vec{x}),\quad\mbox{and}\quad\partial_{t}u(\vec{x},0)=\psi(\vec{x})
\end{equation}
has its solution look like
\begin{equation}
u(\vec{x},t) = \frac{1}{4\pi c^{2}t}\oint_{\bdry B_{ct}(\vec{x})}\psi(\vec{y})\,\D A
+\frac{\partial}{\partial t}\left(\frac{1}{4\pi c^{2}t}\oint_{\bdry B_{ct}(\vec{x})}\phi(\vec{y})\,\D A\right)
\end{equation}
where $B_{r}(\vec{x})$ is the ball of radius $r>0$ centered at $\vec{x}$
(so $\bdry B_{r}(\vec{x})$ is the sphere of radius $r$ centered at $\vec{x}$),
which is known as \define{Kirchoff's Formula} (despite Poisson finding
it first).
\end{subequations}
\end{node} % Scalar wave

\begin{node}[Vector wave]\label{pde:wave-000B}%
The vector wave equation would require using the vector Laplacian,
$\laplacian\vec{u}:=\grad(\divergence\vec{u})-\grad\times(\grad\times\vec{u})$.
In Cartesian coordinates, this coincides with taking the Laplacian of
the components of $\vec{u}$. Then the wave equation looks like
\begin{subequations}
\begin{equation}
\partial_{t}^{2}\vec{u}-c^{2}\laplacian\vec{u}=\vec{0},
\end{equation}
where $c^{2}=\vec{c}\cdot\vec{c}$ is the speed of propagation obtained
from the velocity vector $\vec{c}$ for the wave. We again factorize
\begin{equation}
(\partial_{t}^{2}-\vec{c}\cdot\vec{c}\laplacian)u
=(\partial_{t}-\vec{c}\cdot\grad)(\partial_{t}+\vec{c}\cdot\grad)u=0,
\end{equation}
but a general solution for this problem is too much to hope for. We
typically use the plane-wave solution
$\exp(\pm\I\vec{k}\cdot\vec{x}-\I\omega t)$ where $\omega$ and $\vec{k}$
are constants, then use a linear combination of plane-waves as needed.
\end{subequations}
\end{node} % Vector wave
\end{node} % In three spatial-dimensions

\begin{node}[Causality]\label{pde:wave-0006}%
We can observe that in one-dimension, no part of the wave travels faster
than the speed of propagation. The principle is true in $n>1$ spatial
dimensions, but is obfuscated by distracting calculations.

Observe that if there is an $R>0$ such that for $|x|>R$ we have $\phi=0$
and $\psi=0$, then necessarily $u(x,t)=0$ for $|x|>R+ct$. Really? Well,
all we have to do is look at Eq~\eqref{eq:pde:wave:one-dim:initial-value-problem:soln}
and we see that $\phi(x\pm ct)=0$ for $|x|>R + ct$ and similarly for
$\psi(x\pm ct)=0$. Then $u(x,t)=0$ for $|x|>R+ct$.

\begin{node}\label{pde:wave-000G}%
In particular, if we had a ``time delayed forcing term'', i.e., when we
have $F(x,t)=0$ for $|x|>R$ and $t<t_{1}$, 
so conceptually this would be some ``time delayed perturbation'' to the
region $|x|<R$ starting at time $t_{1}$, then this propagates out no
faster than the speed of propagation (hence the name ``speed of propagation'').
We can verify this by examining the general solution to this problem
\zref{pde:wave-000E}. 
\end{node}
\end{node}