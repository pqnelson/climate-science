% 3
\section{Heat Equation}

\begin{node}[Derivation]\label{pde:heat-0000}%
Let $\temperature(\vec{x},t)$ be the temperature at a point $\vec{x}$ in
a continuous medium and at time $t$. Let $H(t)$ be the amount of heat
(e.g., in calories) contained in a region $V$. Then
\begin{equation*}
H(t)=\int_{V}c\rho\temperature\,\D^{n}x,
\end{equation*}
where $c$ is the specific heat of the medium, and $\rho$ is its mass
density. The change in heat is
\begin{equation*}
\frac{\D H(t)}{\D t}=\int_{V}c\rho\partial_{t}\temperature\,\D^{n}x.
\end{equation*}
Fourier's law says heat flows from hot regions to cold regions
proportionately to the temperature gradient. The only way heat can
change in the region $V$ is by flowing into or out of the region across
its boundary (this is the conservation of energy). Therefore, the change
of heat in $V$ also equals the heat flux across the boundary,
\begin{equation*}
\frac{\D H(t)}{\D t}=\oint_{\bdry V}k\widehat{\vec{n}}\cdot\grad\temperature\,\D A,
\end{equation*}
where $k$ is the thermal conductivity of the material (describing how
easily heat spreads through the medium). By the divergence theorem \zref{vector-calc-0000},
and setting these quantities equal to each other, we find:
\begin{equation}
\int_{V}c\rho\partial_{t}\temperature\,\D^{n}x=\int_{V}\divergence(k\grad\temperature)\,\D^{n}x.
\end{equation}
This is the heat equation. Since the choice of region $V$ was arbitrary,
we therefore must have the integrands be equal
\begin{equation}\label{eq:pde:heat:heat-equation}
\partial_{t}\temperature=\frac{1}{c\rho}\divergence(k\grad\temperature)
\end{equation}
which is how most people recognize the heat equation these days.

Observe, if we pretend $k\grad\temperature$ is a ``flux quantity'', then
we obtain something which resembles a conservation law \zref{pde:first-order-0002}.

\begin{node}[Diffusion]\label{pde:heat-0002}%
If we consider a motionless fluid in a 1-dimensional tube, and we
release dye (or some chemical substance) which is diffusing through the
liquid. We can characterize diffusion using conservation laws, letting
$\rho(t,x)$ be the mass density of the dye at position $x$ of the pipe
at time $t$.

Then the mass of dye in the section of the pipe from $a$ to $b$ is
\begin{subequations}
\begin{equation}
M_{a,b}(t)=\int^{b}_{a}\rho(t,x)\,\D x\quad\mbox{then}\quad%
\frac{\D M_{a,b}(t)}{\D t}=\int^{b}_{a}\partial_{t}\rho(t,x)\,\D x.
\end{equation}
The only way the mass of dye in this section of pipe can change is by
flowing in or out from its endpoints. By Fick's law (the concentration
flux is directly proportional to the spatial gradient of concentration),
\begin{equation}
\frac{\D M_{a,b}(t)}{\D t}=\begin{pmatrix}\mbox{flow}\\\mbox{in}
\end{pmatrix}-\begin{pmatrix}\mbox{flow}\\\mbox{out}
\end{pmatrix}=k\partial_{x}\rho(t,b)-k\partial_{x}\rho(t,a)
\end{equation}
where $k$ is a proportionality constant. Equating these two descriptions
for the rate of change of dye,
\begin{equation}
\int^{b}_{a}\partial_{t}\rho(x,t)\,\D x=k\partial_{x}\rho(t,b)-k\partial_{x}\rho(t,a).
\end{equation}
Differentiating with respect to $b$ gives us the differential equation
\begin{equation}
\partial_{t}\rho=k\partial_{x}^{2}\rho.
\end{equation}
When working in $n>1$ spatial dimensions, we would have
\begin{equation}
\int_{V}\partial_{t}\rho\,\D^{n}x=\oint_{\bdry V}k(\widehat{n}\cdot\grad\rho)\,\D A,
\end{equation}
where $V$ is some compact region and $\bdry V$ is its boundary
surface. Then the divergence theorem \zref{vector-calc-0000} gives us
\begin{equation}
\partial_{t}\rho=\divergence(k\grad\rho).
\end{equation}
Observe this is identical in form as the heat equation we found in
Eq~\eqref{eq:pde:heat:heat-equation}.
\end{subequations}
\end{node}
\end{node}
