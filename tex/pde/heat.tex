% 26
\section{Heat Equation}

\begin{node}[Derivation]\label{pde:heat-0000}%
Let $\temperature(\vec{x},t)$ be the temperature at a point $\vec{x}$ in
a continuous medium and at time $t$. Let $H(t)$ be the amount of heat
(e.g., in calories) contained in a region $V$. Then
\begin{equation*}
H(t)=\int_{V}c\rho\temperature\,\D^{n}x,
\end{equation*}
where $c$ is the specific heat of the medium, and $\rho$ is its mass
density. The change in heat is
\begin{equation*}
\frac{\D H(t)}{\D t}=\int_{V}c\rho\partial_{t}\temperature\,\D^{n}x.
\end{equation*}
Fourier's law says heat flows from hot regions to cold regions
proportionately to the temperature gradient. The only way heat can
change in the region $V$ is by flowing into or out of the region across
its boundary (this is the conservation of energy). Therefore, the change
of heat in $V$ also equals the heat flux across the boundary,
\begin{equation*}
\frac{\D H(t)}{\D t}=\oint_{\bdry V}k\unitnormal\cdot\grad\temperature\,\D A,
\end{equation*}
where $k$ is the thermal conductivity of the material (describing how
easily heat spreads through the medium). By the divergence theorem \zref{vector-calc-0000},
and setting these quantities equal to each other, we find:
\begin{equation}
\int_{V}c\rho\partial_{t}\temperature\,\D^{n}x=\int_{V}\divergence(k\grad\temperature)\,\D^{n}x.
\end{equation}
This is the heat equation. Since the choice of region $V$ was arbitrary,
we therefore must have the integrands be equal
\begin{equation}\label{eq:pde:heat:heat-equation}
\partial_{t}\temperature=\frac{1}{c\rho}\divergence(k\grad\temperature)
\end{equation}
which is how most people recognize the heat equation these days.

Observe, if we pretend $k\grad\temperature$ is a ``flux quantity'', then
we obtain something which resembles a conservation law \zref{pde:first-order-0002}.

\begin{node}[Diffusion]\label{pde:heat-0002}%
If we consider a motionless fluid in a 1-dimensional tube, and we
release dye (or some chemical substance) which is diffusing through the
liquid. We can characterize diffusion using conservation laws, letting
$\rho(t,x)$ be the mass density of the dye at position $x$ of the pipe
at time $t$.

Then the mass of dye in the section of the pipe from $a$ to $b$ is
\begin{subequations}
\begin{equation}
M_{a,b}(t)=\int^{b}_{a}\rho(t,x)\,\D x\quad\mbox{then}\quad%
\frac{\D M_{a,b}(t)}{\D t}=\int^{b}_{a}\partial_{t}\rho(t,x)\,\D x.
\end{equation}
The only way the mass of dye in this section of pipe can change is by
flowing in or out from its endpoints. By Fick's law (the concentration
flux is directly proportional to the spatial gradient of concentration),
\begin{equation}
\frac{\D M_{a,b}(t)}{\D t}=\begin{pmatrix}\mbox{flow}\\\mbox{in}
\end{pmatrix}-\begin{pmatrix}\mbox{flow}\\\mbox{out}
\end{pmatrix}=k\partial_{x}\rho(t,b)-k\partial_{x}\rho(t,a)
\end{equation}
where $k$ is a proportionality constant. Equating these two descriptions
for the rate of change of dye,
\begin{equation}
\int^{b}_{a}\partial_{t}\rho(x,t)\,\D x=k\partial_{x}\rho(t,b)-k\partial_{x}\rho(t,a).
\end{equation}
Differentiating with respect to $b$ gives us the differential equation
\begin{equation}
\partial_{t}\rho=k\partial_{x}^{2}\rho.
\end{equation}
When working in $n>1$ spatial dimensions, we would have
\begin{equation}
\int_{V}\partial_{t}\rho\,\D^{n}x=\oint_{\bdry V}k(\widehat{n}\cdot\grad\rho)\,\D A,
\end{equation}
where $V$ is some compact region and $\bdry V$ is its boundary
surface. Then the divergence theorem \zref{vector-calc-0000} gives us
\begin{equation}
\partial_{t}\rho=\divergence(k\grad\rho).
\end{equation}
Observe this is identical in form as the heat equation we found in
Eq~\eqref{eq:pde:heat:heat-equation}.
\end{subequations}
\end{node}
\end{node}

\begin{node}[One-dimensional diffusion over an interval]\label{pde:heat-0006}%

\begin{theorem}[Maximum principle]\label{pde:heat-0003}%
If $u(x,t)$ satisfies the heat equation in a rectangle $0\leq x\leq\ell$,
$0\leq t\leq t_{\text{max}}$ in space-time,  then the maximum value of
$u(x,t)$ must be at either $t=0$ or $x\in\{0,\ell\}$.
\end{theorem}

\begin{node}[Dirichlet problem]\label{pde:heat-0004}%
The Dirichlet problem for the heat equation in one-dimension is:
\begin{equation}
\begin{split}
\partial_{t}u - k\partial_{x}^{2}u &=f(x,t)\\
u(x,0) &= \phi(x)\\
u(0,t)&=g(t)\\
u(\ell,t) &= h(t),
\end{split}
\end{equation}
where $0<x<\ell$ and $t>0$, and $f$, $g$, $h$, and $\phi$ are given
functions.

\begin{theorem}[Uniqueness]\label{pde:heat-0005}%
If a solution exists for the Dirichlet problem for the heat equation,
then it is unique.
\end{theorem}

\begin{proof}[Proof (Uniqueness via maximum principle)]
Let $u_{1}(x,t)$ and $u_{2}(x,t)$ be two solutions. Let $w=u_{1}-u_{2}$.
Then $\partial_{t}w-k\partial_{x}^{2}w=0$, $w(x,0)=0$, $w(0,t)=0$,
$w(\ell,t)=0$. Let $t_{\text{max}}>0$ be arbitrary.

If $w(x,t)$ has its maximum on the rectangle, then it must be on the
sides (where it vanishes). So $w(x,t)\leq0$. The same argument for the
minimum shows $w(x,t)\geq0$. Therefore $w(x,t)=0$ identically, and this
implies $u_{1}=u_{2}$ for all $t>0$.
\end{proof}

\begin{proof}[Proof 2 (Uniqueness via energy method)]
Let $u_{1}(x,t)$ and $u_{2}(x,t)$ be two solutions. Let $w=u_{1}-u_{2}$.
We have
\begin{calculation}
  0
\step{zero times anything is zero}
  0\cdot w
\step{since $\partial_{t}w-k\partial_{x}^{2}w=0$}
  (\partial_{t}w-k\partial_{x}^{2}w)\cdot(w)
\step{using power rule for differentiation}
  \partial_{t}(w^{2}/2) + \partial_{x}(-kw\partial_{x}w)+k(\partial_{x}w)^{2},
\end{calculation}
then integrating over the integral $0<x<\ell$ gives us
\begin{equation*}
0 = \left.\int^{\ell}_{0}\partial_{t}(w^{2}/2)\,\D x - kw\partial_{x}w\right|^{x=\ell}_{x=0}+k\int^{\ell}_{0}(\partial_{x}w)^{2}\,\D x.
\end{equation*}
Now the first term we can rewrite, using the boundary conditions $w=0$
at $x=0$ and also at $x=\ell$,
\begin{equation*}
\frac{\D}{\D t}\int^{\ell}_{0}\frac{1}{2}[w(x,t)]^{2}\,\D x
=-k\int^{\ell}_{0}[\partial_{x}w(x,t)]^{2}\,\D x\leq0.
\end{equation*}
But this means $\int w^{2}\D x$ is decreasing over time, which means
\begin{equation}
\int^{\ell}_{0}[w(x,t)]^{2}\,\D x\leq\int\int^{\ell}_{0}[w(x,0)]^{2}\,\D x,
\end{equation}
for all $t\geq0$. The right-hand side vanishes because the initial
conditions for $u_{1}$ and $u_{2}$ are identical, so $\int[w(x,t)]^{2}\D x=0$
for all $t>0$. Therefore $w=0$, hence $u_{1}=u_{2}$ for all $t\geq0$.  
\end{proof}
\end{node} % Dirichlet problem
\end{node} % One-dimensional diffusion over an interval

\begin{node}[Diffusion on real line]\label{pde:heat-0007}%
Consider now the problem
\begin{subequations}\label{eq:pde:heat:diffusion-on-real-line}
\begin{align}
\partial_{t}u &= k\partial_{x}^{2}u\\
u(x,0) &= \phi(x),
\end{align}
\end{subequations}
where $-\infty<x<+\infty$ and $0<t<\infty$.

\begin{node}[Invariance properties]\label{pde:heat-0008}%
We can observe:
\begin{enumerate}
\item If $u(x,t)$ is a solution and $y$ is a fixed real number, then
  $u(x-y,t)$ is another solution;
\item If $u(x,t)$ is a solution, then any derivative of $u$ is a solution;
\item Any linear combination of solutions are another solution;
\item An integral of solutions is a solution and so is
\begin{equation*}
v(x,t)=\int^{\infty}_{-\infty}u(x-y,t)g(y)\,\D y
\end{equation*}
for any function $g(y)$ (as long as the integral is defined and converges);
\item Dilation property: If $u(x,t)$ is a solution, then so is $u(\sqrt{\lambda}x,\lambda t)$
  for any real constant $\lambda>0$ --- this can be obtained by the
  chain rule, setting $v(x,t)=u(\sqrt{\lambda}x,\lambda t)$ and so
  $\partial_{t}v=[\partial_{t}(\lambda t)]\partial_{t}u=\lambda\partial_{t}u$
  and
  $\partial_{x}v=[\partial_{x}(\sqrt{\lambda}x)]\partial_{x}u=\sqrt{\lambda}\partial_{x}u$
  and so $\partial_{x}^{2}v=\lambda\partial_{x}^{2}u$.
\end{enumerate}
\end{node} % Invariance properties

\begin{node}[Solution]\label{pde:heat-0009}%
We can now find a solution satisfying the particular initial condition
$W(x,0)=1$ for $x>0$, and $W(x,0)=0$ for $x<0$. This can be done in
several steps.

\begin{node}[Step 1]\label{pde:heat-000A}%
We expect, by the dilation property of a solution to the heat equation
on the real line, that it is reasonable to look for $W(x,t)$ of the form
\begin{equation}
W(x,t) = g(s)\quad\mbox{where}\quad s=\frac{x}{\sqrt{4kt}},
\end{equation}
and $g$ is a function of a single variable.

The dilation takes $x/\sqrt{t}$ to $(\sqrt{\lambda}x)/\sqrt{\lambda t}=x/\sqrt{t}$.
Therefore it makes sense to let $s\propto x/\sqrt{t}$ up to some nonzero
positive constant.
\end{node} % step 1

\begin{node}[Step 2]\label{pde:heat-000B}%
Now we compute the derivatives of $W(x,t)$ in terms of $g(s)$:
\begin{equation*}
\partial_{t}W=g'(s)\partial_{t}s=\frac{-1}{2t}\frac{x}{\sqrt{4kt}}g'(s)
\end{equation*}
\begin{equation*}
\partial_{x}W=g'(s)\partial_{x}s=\frac{1}{\sqrt{4kt}}g'(s)
\end{equation*}
\begin{equation*}
\partial_{x}^{2}W=\frac{1}{4kt}g'(s)
\end{equation*}
Then combining this all together,
\begin{equation*}
0 = \partial_{t}W-k\partial_{x}^{2}W=\frac{1}{t}\left(\frac{-1}{2}sg'(s)-\frac{1}{4}g''(s)\right).
\end{equation*}
Thus we obtain
\begin{equation}
g''(s) + 2sg'(s)=0.
\end{equation}
This can be integrated (multiply through by $\exp(s^{2})$) to find
$g'(s)=c_{1}\exp(-s^{2})$, and we can integrate again to obtain:
\begin{equation}
W(x,t)=g(s)=c_{1}\int\E^{-s^{2}}\D s + c_{2}.
\end{equation}
\end{node} % step 2

\begin{node}[Step 3]\label{pde:heat-000C}%
We now have
\begin{equation}
W(x,t)=g(s)=c_{1}\int^{x/\sqrt{4kt}}_{0}\E^{-s^{2}}\D s + c_{2}.
\end{equation}
We determine the constants of integration by taking the limit as
$t\to0^{+}$ from the right. If $x>0$, then
\begin{equation*}
1=\lim_{t\to0^{+}}W(x,t)=c_{1}\frac{\sqrt{\pi}}{2}+c_{2},
\end{equation*}
and if $x<0$ then
\begin{equation*}
0=\lim_{t\to0^{+}}W(x,t)=-c_{1}\frac{\sqrt{\pi}}{2}+c_{2},
\end{equation*}
which implies $c_{1}=1/\sqrt{\pi}$ and $c_{2}=1/2$. Therefore,
\begin{equation}
W(x,t) =
\frac{1}{2}+\frac{1}{\sqrt{\pi}}\int^{x/\sqrt{4kt}}_{0}\E^{-s^{2}}\,\D s,
\end{equation}
for $t>0$.
\end{node} % step 3

\begin{node}[Step 4]\label{pde:heat-000D}%
Now we define $S:=\partial_{x}W$, which will also be a solution to the
heat equation on the line. For any integrable function $\phi$, we also
have
\begin{equation}
u(x,t)=\int^{\infty}_{-\infty}S(x-y,t)\phi(y)\,\D y
\end{equation}
for $t>0$. Then $u$ is the unique solution to the initial value problem 
Eq~\eqref{eq:pde:heat:diffusion-on-real-line}.

We have explicitly,
\begin{equation}
S(x,t) = \frac{1}{(4\pi kt)^{1/2}}\exp\left(\frac{-x^{2}}{4kt}\right)
\end{equation}
for $t>0$. This is the \define{Heat Kernel}. Further, in $\RR^{n}$, the
heat equation has its corresponding heat kernel
\begin{equation}
S(\vec{x},t) = \frac{1}{(4\pi kt)^{n/2}}\exp\left(\frac{-\|\vec{x}\|^{2}}{4kt}\right)
\end{equation}
and can be used similarly to find solutions to the initial value problem
for the Heat equation in $\RR^{n}$.
\end{node} % step 4

\begin{definition}\label{pde:heat-000E}%
Let $n\in\NN$, let $k>0$ be a constant.
We define the \define{Heat Kernel} for $\RR^{n}$ to be
\begin{equation*}
S(\vec{x},t) = \frac{1}{(4\pi kt)^{n/2}}\exp\left(\frac{-\|\vec{x}\|^{2}}{4kt}\right).
\end{equation*}
\end{definition}
\end{node} % solution

\begin{node}[Solution with sources]\label{pde:heat-000F}%
When we have
\begin{equation*}
\partial_{t}u = k\partial_{x}^{2}u+f(x,t)
\end{equation*}
on $x\in\RR$, $t>0$ with initial condition $u(x,0)=\phi(x)$, we find its
solution: 
\begin{equation}
\begin{split}
u(x,t) &=\int^{\infty}_{-\infty}S(x-y,t)\phi(y)\,\D y\\
&\qquad+\int^{t}_{0}\int^{\infty}_{-\infty}S(x-y,t-s)f(y,s)\,\D y\,\D s.
\end{split}
\end{equation}

\begin{proof}
We have \textbf{assuming} $\phi=0$ for simplicity,
\begin{calculation}
  \partial_{t}u
\step{differentiating under the integral sign \zref{calculus:integral-000G}}
  \lim_{s\to t}\int^{\infty}_{-\infty}S(x-y,t-s)f(y,s)\,\D y
  + \int^{t}_{0}\int^{\infty}_{-\infty}\partial_{t}S(x-y,t-s)f(y,s)\,\D y\,\D s
\step{since $S$ satisfies the heat equation}
  \lim_{\varepsilon\to 0}\int^{\infty}_{-\infty}S(x-y,\varepsilon)f(y,t)\,\D y
  + \int^{t}_{0}\int^{\infty}_{-\infty}k\partial_{x}^{2}S(x-y,t-s)f(y,s)\,\D y\,\D s
\step{using initial condition of $S$}
  f(x,t)
  + \int^{t}_{0}\int^{\infty}_{-\infty}k\partial_{x}^{2}S(x-y,t-s)f(y,s)\,\D y\,\D s
\step{pulling out the derivatives}
  f(x,t)
  + k\partial_{x}^{2}\int^{t}_{0}\int^{\infty}_{-\infty}S(x-y,t-s)f(y,s)\,\D y\,\D s
\step{folding back the definition $u=\iint S(x-y,t-s)f(y,s)\,\D y\,\D s$}
  f(x,t)
  + k\partial_{x}^{2}u(x,t)
\end{calculation}
Then by the superposition principle \zref{pde-000J}, the result follows.
\end{proof}
\end{node} % Solution with sources
\end{node} % Diffusion on real line

\begin{node}[Diffusion on the half-line]\label{pde:heat-000G}%
Consider now the problem in $0<x<\infty$ and $0<t<\infty$ with
\begin{equation*}
\partial_{t}u = k\partial_{x}^{2}u
\end{equation*}

\begin{node}[Solution to Dirichlet problem]\label{pde:heat-000H}%
Consider the initial conditions:
\begin{subequations}
\begin{align}
u(x,0) &= \phi(x)\\
u(0,t) &= 0.
\end{align}
\end{subequations}
The trick is to extend this to the real line by taking as initial data
on the real line
$\phi_{\text{odd}}(-x)=-\phi(x)$ and $\phi_{\text{odd}}(x)=\phi(x)$ for $x>0$
and $\phi_{\text{odd}}(0)=0$. Taking the solution for this extended problem,
\begin{subequations}
\begin{align}
  u_{\text{ext}}(x,t)
  &=\int^{\infty}_{-\infty}S(x-y,t)\phi_{\text{odd}}(y)\,\D y\\
  &=\int^{\infty}_{0}S(x-y,t)\phi(y)\,\D y
  -\int^{0}_{-\infty}S(x-y,t)\phi(-y)\,\D y\\
  &=\int^{\infty}_{0}[S(x-y,t) - S(x+y,t)]\phi(y)\,\D y
\end{align}
Then we see the solution to our problem on the half-line is:
\begin{equation}
\boxed{u(x,t) = \frac{1}{\sqrt{4\pi kt}}\int^{\infty}_{0}[\E^{-(x-y)^{2}/4kt}-\E^{-(x+y)^{2}/4kt}]\phi(y)\,\D y.}
\end{equation}
\end{subequations}
\end{node} % Solution to Dirichlet problem

\begin{node}[Solution with Neumann boundary conditions]\label{pde:heat-000I}%
Now consider the Neumann boundary conditions
\begin{subequations}
\begin{align}
u(x,0) &= \phi(x)\\
\partial_{x}u(0,t) &= 0.
\end{align}
The trick in this case is to work with the even extension of $\phi(x)$,
\begin{equation}
\phi_{\text{even}}(x) = \begin{cases}\phi(x) &\mbox{for }x\geq0\\
+\phi(-x) &\mbox{for }x\leq0.
\end{cases}
\end{equation}
Now using similar reasoning as for the Dirichlet problem on the
half-line, we find the solution for the Neumann boundary condition:
\begin{equation}
\boxed{u(x,t) = \frac{1}{\sqrt{4\pi kt}}\int^{\infty}_{0}[\E^{-(x-y)^{2}/4kt}+\E^{-(x+y)^{2}/4kt}]\phi(y)\,\D y.}
\end{equation}
\end{subequations}
\end{node} % Solution with Neumann boundary conditions

\begin{node}[Homogeneous Dirichlet boundary condition with source]\label{pde:heat-000J}%
Consider now the partial differential equation, with $0<x<\infty$ and $0<t<\infty$,
\begin{subequations}
\begin{align}
\partial_{t}u &= k\partial_{x}^{2}u + f(x,t)\\
u(x,0) &= 0\\
u(0,t) &= 0,
\end{align}
\end{subequations}
Then we have the odd extension $f_{\text{odd}}(-x,t)=-f(x,t)$ and
$f_{\text{odd}}(x,t)=f(x,t)$ for $x>0$, and $f_{\text{odd}}(0,t)=0$,
which can be used to find the solution for the entire real line. This is
used to obtain a solution on the half-line as
\begin{equation}
\boxed{u(x,t) = \int^{t}_{0}\int^{\infty}_{0}(S(x-y,t-s)-S(x+y,t-s))f(y,s)\,\D y\,\D s.}
\end{equation}
If we want to add some initial condition $u(x,0)=\phi(x)$, then we just
need to add to our result the solution for the Dirichlet problem with no
source and initial condition \zref{pde:heat-000H}, which would be
\begin{equation}
\begin{split}
u(x,t) &= \int^{t}_{0}\int^{\infty}_{0}(S(x-y,t-s)-S(x+y,t-s))f(y,s)\,\D y\,\D s\\
&\phantom{=}\quad+\int^{\infty}_{0}[S(x-y,t)-S(x+y,t)]\phi(y)\,\D y.
\end{split}
\end{equation}
\end{node} % Homogeneous Dirichlet boundary condition with source

\begin{node}[General solution for Dirichlet boundary condition]\label{pde:heat-000L}%
If we have an inhomogeneous Dirichlet condition, a source term, and
initial data:
\begin{subequations}
\begin{align}
\partial_{t}u &= k\partial_{x}^{2}u + f(x,t)\\
u(x,0) &= \phi(x)\\
u(0,t) &= h(t).
\end{align}
\end{subequations}
We can solve this by considering $v(x,t)=u(x,t)-h(t)$ with a source $f(x,t)+h'(t)$ and
Dirichlet condition $v(0,t)=0$ and initial condition $v(x,0)=\phi(x)-h(0)$.
Explicitly, this is:
\begin{equation}
\begin{split}
u(x,t) &= \int^{t}_{0}\int^{\infty}_{0}(S(x-y,t-s)-S(x+y,t-s))(f(y,s)-h'(s))\,\D y\,\D s\\
&\phantom{=}\quad+\int^{\infty}_{0}[S(x-y,t)-S(x+y,t)]\phi(y)\,\D y+h(t).
\end{split}
\end{equation}
\end{node} % General solution for Dirichlet boundary condition

\begin{node}[Homogeneous Neumann boundary condition with source]\label{pde:heat-000K}%
Consider now the partial differential equation, with $0<x<\infty$ and $0<t<\infty$,
\begin{subequations}
\begin{align}
\partial_{t}u &= k\partial_{x}^{2}u + f(x,t)\\
u(x,0) &= 0\\
\partial_{x}u(0,t) &= 0,
\end{align}
\end{subequations}
Then by considering the even extension of $f$, we obtain a solution to
diffusion on the full real line, and recover a solution for the
homogeneous Neumann problem with source as:
\begin{equation}
\boxed{u(x,t) = \int^{t}_{0}\int^{\infty}_{0}(S(x-y,t-s)+S(x+y,t-s))f(y,s)\,\D y\,\D s.}
\end{equation}
If we have some initial data $u(x,0)$, then we can add to this the
solution for the source-free Neumann heat equation on the half-line
\zref{pde:heat-000I}.
\end{node} % Homogeneous Neumann boundary condition with source

\begin{node}[Inhomogeneous Neumann boundary condition with source]\label{pde:heat-000M}%
If we have an inhomogeneous Neumann boundary condition
$\partial_{x}u(0,t)=h(t)$ for the heat equation on the half-line with no
initial data $u(x,0)=0$ and a source term $f(x,t)$, then we consider
$v(x,t)=u(x,t)-xh(t)$ and the problem expressed using $v(x,t)$ is
equivalent to the homogeneous Neumann boundary condition for $v$ with
its source term being $xh'(t)$. This would give us 
\begin{equation}
\begin{split}
u(x,t) &= \int^{t}_{0}\int^{\infty}_{0}(S(x-y,t-s)+S(x+y,t-s))(f(y,s)+yh'(s))\,\D y\,\D s\\
&\phantom{=}\quad+xh(t).
\end{split}
\end{equation}
If we wanted to add some nonzero initial data $u(x,0)=\phi(x)$, then we add
the solution for the homogeneous source-free Neumann boundary conditions
for the heat equation on the half-line \zref{pde:heat-000I}.
\end{node} % Inhomogeneous Neumann boundary condition with source

\begin{node}[General solution for Neumann boundary condition]\label{pde:heat-000N}%
Combining all these considerations, we have the Neumann boundary
conditions for the heat equation on the half-line with source and
initial data
\begin{subequations}
\begin{align}
\partial_{t}u &= k\partial_{x}^{2}u + f(x,t)\\
u(x,0) &= \phi(x)\\
\partial_{x}u(0,t) &= h(t),
\end{align}
\end{subequations}
This has its general solution look like
\begin{equation}
\begin{split}
u(x,t) &= \int^{\infty}_{0}(S(x-y,t-s)+S(x+y,t-s))\phi(y)\,\D y\\
&\phantom{=}\quad+\int^{t}_{0}\int^{\infty}_{0}(S(x-y,t-s)+S(x+y,t-s))(f(y,s)+yh'(s))\,\D y\,\D s\\
&\phantom{=}\quad+xh(t)
\end{split}
\end{equation}
which combines the homogeneous Neumann boundary condition with initial data \zref{pde:heat-000I}
with the inhomogeneous Neumann boundary condition with source but no
initial data \zref{pde:heat-000M}.
\end{node} % General solution for Neumann boundary condition

\end{node} % Diffusion on the half-line


\begin{node}[Navier--Stokes and heat equation]\label{pde:heat-000O}%
We can note that the Navier--Stokes equation describing a fluid with
viscosity (``friction'' between fluid parcels) looks like the heat
equation plus some convective term
\begin{equation*}
\partial_{t}\vec{u}+(\vec{u}\cdot\grad)\vec{u}=\nu\laplacian\vec{u}-\frac{1}{\rho}\grad p.
\end{equation*}
The Stokes system discards the convective term $(\vec{u}\cdot\grad)\vec{u}\to0$
and we obtain a system of heat equations with source terms. For this
reason, the heat equation is especially important in fluids, and is why
I spent so much time analyzing different boundary conditions for the
heat equation.
\end{node} % Navier--Stokes and heat equation

\begin{node}[Causality]\label{pde:heat-000P}%
We can illustrate problems the heat equation experiences with causality
by examining the situation on the real line
when we have a source term $f(x,t)=0$ for $t<t_{0}$ and $f(x,t)\neq0$
for $t>t_{0}$ and $|x|<R$. Then we know the solution \zref{pde:heat-000F}
looks like
\begin{equation*}
\begin{split}
u(x,t) &=\int^{\infty}_{-\infty}S(x-y,t)\phi(y)\,\D y\\
&\qquad+\int^{t}_{0}\int^{\infty}_{-\infty}S(x-y,t-s)f(y,s)\,\D y\,\D s.
\end{split}
\end{equation*}
The instant that $t>t_{0}$, all of $u(x,t)$ will be impacted by it, due
to the term involving integrating over the source term. In some $\Delta t>0$
after $t_{0}$, every point in space will be affected by a contribution of
\begin{equation*}
\begin{split}
\Delta u(x,t_{0}+\Delta t)
&=\int^{t_{0}+\Delta t}_{t_{0}}\int^{\infty}_{-\infty}S(x-y,t-s)f(y,s)\,\D y\,\D s\\
&\approx\Delta t\int^{\infty}_{-\infty}S(x-y,\Delta t)f(y,t_{0})\,\D y\neq0.
\end{split}
\end{equation*}
This means that the heat equation does not respect causality:
information has transmitted instantaneously to every point of the real
line.
\end{node}

\begin{puzzle}[Heat equation respecting causality]
Is there a version of the heat equation which respects causality? I am
aware of a relativistic version of the heat equation, but it appears to
be a ``minimally modified'' heat equation and has difficulties matching
empirical results (I suspect because it is ``minimally modified'' rather
than something coherently derived from first principles).
\end{puzzle}
