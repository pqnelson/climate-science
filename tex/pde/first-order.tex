% 3
\section{First-order equations}

% https://www.ucl.ac.uk/~ucahhwi/LTCC/sectionA-firstorder.pdf

\begin{node}\label{pde-0002}%
A linear first-order partial differential equation in, say, two
variables $x$ and $y$ looks like
\[a(x,y)\partial_{x}u+b(x,y)\partial_{y}u+c(x,y)u=f(x,y).\]
Let us consider this more carefully in the homogeneous and inhomogeneous
cases.

\begin{node}[Homogeneous]\label{pde-0003}%
We begin with
\[a(x,y)\partial_{x}u+b(x,y)\partial_{y}u=0.\]
If we supposed that we had a trajectory $(x(t),y(t))$ and considered
\[\frac{\D}{\D t}u(x(t),y(t))=\frac{\D x}{\D t}\partial_{x}u+\frac{\D y}{\D t}\partial_{y}u\]
we can identify the coefficients of $\partial_{x}u$ as equal (and
similarly for $\partial_{y}u$) giving us the equations
\[\frac{\D x}{\D t}=a(x(t),y(t)),\quad\mbox{and}\quad\frac{\D y}{\D t}=b(x(t),y(t)).\]

\begin{example}\label{pde-0005}%
Consider the wave moving at constant velocity $c$:
\[\partial_{t}u+c\partial_{x}u=0.\]
We can easily change variables to
\[\xi=x+ct,\quad\mbox{and}\quad\eta=x-ct.\]
Then we find (thanks to our beloved chain rule)
\[\partial_{x}=\partial_{x}\xi\partial_{\xi}+\partial_{x}\eta\partial_{\eta}=\partial_{\xi}+\partial_{\eta},\]
and similarly
\[\partial_{t}=\partial_{t}\xi\partial_{\xi}+\partial_{t}\eta\partial_{\eta}=c\partial_{\xi}-c\partial_{\eta}.\]
Then we find our wave equation simplifies to
\[2c\partial_{\xi}u=0.\]
Integration yields the solution $u=f(\eta)$ for any $C^{1}$ function $f$.

How did we determine the change of coordinates? Well, one way to view it
is that we parametrized $x=x(s)$ and $t=t(s)$, if we moved along the
line formed by $x-ct=\mbox{constant}$, then we would have that the
$s$ derivative of $u(x(s),t(s))$ be constant,
\[\frac{\D u(x(s),t(s))}{\D s}=0.\]
This is the principle underpinning the method of characteristics.
\end{example}

\begin{example}\label{pde-0006}%
If we have variable speed,
\[\partial_{t}u+c(x,t)\partial_{x}u=0.\]
Now we want to find curves along which $u$ is constant. We begin by
supposing we have a curve $x=x(s)$ and $t=t(s)$. Then, using the chain
rule, we have
\[\frac{\D u(x(s),t(s))}{\D s}=\partial_{t}u\frac{\D t(s)}{\D s} + \partial_{x}u\frac{\D x(s)}{\D s}.\]
Matching the coefficients of partial derivatives of $u$ with those in
the original partial differential equation
\[\partial_{t}u\frac{\D t(s)}{\D s} + \partial_{x}u\frac{\D x(s)}{\D s} =\partial_{t}u+c(x,t)\partial_{x}u.\]
We have
\[\frac{\D t(s)}{\D s}=1,\quad\mbox{and}\quad\frac{\D x(s)}{\D s}=c(x(s),s).\]
We can solve these equations (and, indeed, implicitly did so in setting
up $\D x/\D s$).
\end{example}

\begin{example}\label{pde-0004}%
Consider the partial differential equation
\[\frac{2\sin^{2}(\theta)}{\cos(\theta)}\partial_{\theta}u-\frac{\sin(2\phi)}{\cos(2\phi)}\partial_{\phi}u=0.\]
We have the parametrization $(\theta(t),\phi(t))$ give us the
differential equations
\[\frac{\D\theta}{\D t}=\frac{2\sin^{2}(\theta)}{\cos(\theta)},\quad\mbox{and}\quad\frac{\D\phi}{\D t}=-\frac{\sin(2\phi)}{\cos(2\phi)}.\]
We can solve these equations
\[\int\frac{\cos(\theta)}{\sin^{2}(\theta)}\D\theta=2\int\D t\]
which is
\[\frac{-1}{\sin(\theta)}+\frac{1}{\sin(\theta_{0})}=2t-2t_{0}.\]
Similarly,
\[-\int\frac{\cos(2\phi)}{\sin(2\phi)}\D\phi=\int\D t,\]
which is (changing variables to $u=\sin(2\phi)$, $\D u=2\cos(2\phi)\,\D\phi$,
so $\cos(2\phi)\,\D\phi=\frac{1}{2}\D u$),
\[\ln(\sin(2\phi)/\sin(2\phi_{0}))=2t_{0}-2t \implies\sin(2\phi)=\sin(2\phi_{0})\exp(2t_{0}-2t).\]
We want to isolate the constants of integration on one side of these
results.

Now taking $t-t_{0}$ from our first solution and plugging it into our
second yields
\[\sin(2\phi)=\sin(2\phi_{0})\exp\left(\frac{1}{\sin(\theta)}-\frac{1}{\sin(\theta_{0})}\right).\]
By direct calculation
\begin{calculation}
\sin(2\phi)=\sin(2\phi_{0})\exp\left(\frac{1}{\sin(\theta)}-\frac{1}{\sin(\theta_{0})}\right)
\step[\equiv]{take log of both sides}
\ln(\sin(2\phi))=\ln(\sin(2\phi_{0}))+\frac{1}{\sin(\theta)}-\frac{1}{\sin(\theta_{0})}
\step[\equiv]{subtracting $1/2\sin(\theta)$ from both sides}
\ln(\sin(2\phi))-\frac{1}{\sin(\theta)}=\ln(\sin(2\phi_{0}))-\frac{1}{\sin(\theta_{0})}
\end{calculation}
Observe the right-hand side is a constant, so if we then write
\[u=F\left(\ln(\sin(2\phi))-\frac{1}{\sin(\theta)}\right)\]
for any arbitrary $C^{1}$ function $F$, then we have obtained a generic
solution to our differential equation.
\end{example}

\begin{node}\label{pde-0007}%
Generalizing to more than 2 dimensions is straightforward, so I'm going
to skip it.
\end{node}
\end{node} % homogeneous

\begin{node}[Inhomogeneous]\label{pde-0008}%

\begin{example}\label{pde-0009}%
Consider the partial differential equation
\[\partial_{t}u + 2xt\partial_{x}u=u,\]
with initial condition $u(x,t=0)=x$. We find the characteristics,
\[\frac{\D t(s)}{\D s}=1\implies t=s-s_{0},\]
and
\begin{calculation}
  \frac{\D x(s)}{\D s}=2x\cdot(s-s_{0})
\step[\equiv]{algebra}
  \frac{1}{2x(s)}\frac{\D x(s)}{\D s}=s-s_{0}
\step[\equiv]{integrate both sides from $t=0$, i.e., $s=s_{0}$}
  \frac{1}{2}\ln(x/x_{0})=\frac{1}{2}s^{2}-s_{0}s+\frac{1}{2}s_{0}^{2}=\frac{1}{2}(s-s_{0})^{2}
\step[\equiv]{multiply both sides by 2, then exponentiating both sides}
  x=x_{0}\exp\bigl((s-s_{0})^{2}\bigr)
\end{calculation}
Now we have
\[\frac{\D u(x(s),t(s))}{\D s}=\partial_{t}u+\frac{\D x(s)}{\D s}\partial_{x}u=u\implies u=F(x_{0})\E^{s}.\]
Since $x=x_{0}\exp(t^{2})$ or equivalently $x_{0}=x\exp(-t^{2})$, we
have (up to some irrelevant multiplicative constant which can be
absorbed into $F$) obtained $u=F(x_{0})\E^{t}$, and therefore
\[u=F(x\E^{-t^{2}})\exp(t).\]
When we apply the initial conditions that at $t=0$ we have
$u(x(0),t=0)=x_{0}$ and $u=F(x_{0})$ which means
$F(x_{0})=x_{0}$. Therefore we find
\[u(x,t)=x\exp(t-t^{2}).\]
\end{example}
\end{node} % inhomogeneous
\end{node} % linear first-order

\begin{node}[Nonlinear first-order]\label{pde-000A}%
Nonlinear first-order partial differential equations come in a variety
of degrees of ``nonlinear-ness''. For example, a \define{Semilinear}
first-order PDE has linear coefficients but a nonlinear inhomogeneous
term
\begin{equation}
a(x,y)\partial_{x}u+b(x,y)\partial_{y}u=f(x,y,u).
\end{equation}
A \define{Quasilinear} first-order PDE would look like
\begin{equation}
a(x,y,u)\partial_{x}u+b(x,y,u)\partial_{y}u=f(x,y,u).
\end{equation}
The difficulty that we encounter with nonlinear first-order PDEs is that
the method of characteristics produces an implicit equation for $u$
rather than an explicit one.
\end{node}


\begin{node}[Conservation Laws]\label{pde:first-order-0000}%
\begin{node}[Example: mass]\label{pde:first-order-0001}%
Let $\rho(t,x)$ represent mass density for a one-dimensional continuous
body (a fluid, say). At time $t$, the total mass lying in the interval
$a\leq x\leq b$ is calculated by
\begin{subequations}\label{eq:pde:first-order:conservation:mass-example}
\begin{equation}
M_{a,b}(t)=\int^{b}_{a}\rho(t,x)\,\D x.
\end{equation}
This assumes that mass is not created or destroyed in this region (we
can model that later).

There may be mass flowing into the region, and mass flowing out of the
region. This is described by the \define{Flux} $j(t,x)$ and we would
have $j(t,a)$ describe the amount flowing into the region, $-j(t,b)$ the
amount flowing out of the region, so this combines to give us
\begin{equation}
\frac{\D}{\D t}M_{a,b}(t)=j(t,a)-j(t,b)=\int^{b}_{a}-\partial_{x}j(t,x)\,\D x
\end{equation}
The total change over time of mass in this region
would be
\begin{calculation}
  \frac{\D}{\D t} M_{a,b}(t)
\step{using Eq~\eqref{eq:pde:first-order:conservation:mass-example}}
\frac{\D}{\D t}\int^{b}_{a}\rho(t,x)\,\D x
\step{since the boundaries are fixed with respect to time}
\int^{b}_{a}\partial_{t}\rho(t,x)\,\D x
\step{using Eq~\eqref{eq:pde:first-order:conservation:mass-example}}
\int^{b}_{a}-\partial_{x}j(x,t)\,\D x
\end{calculation}
Then rearranging these last two results gives us
\begin{equation}
\int^{b}_{a}\partial_{t}\rho(t,x)-\partial_{x}j(t,x)\,\D x=0.
\end{equation}
Since this is for any arbitrary interval $[a,b]$, this means the
integrand must vanish, which is precisely the one-dimensional
differential form of the conservation of mass
\begin{equation}
\partial_{t}\rho(t,x)-\partial_{x}j(t,x)=0.
\end{equation}
This is generalized to higher dimensions by turning $\vec{j}(t,\vec{x})$
into a vector field, the flow into and out of the region is precisely
the integral of $\vec{j}\cdot\widehat{\vec{n}}$ (where
$\widehat{\vec{n}}$ is the outward-facing unit normal vector on the
boundary of the region), then the divergence theorem \zref{vector-calc-0000}
gives us the direct generalization to the 1-dimensional conservation law:
\begin{equation}
\partial_{t}\rho(t,x)-\divergence\vec{j}(t,\vec{x})=0.
\end{equation}
\end{subequations}
\end{node}

\begin{node}[In general]\label{pde:first-order-0002}%
We generalize from mass to an arbitrary quantity $q=q(t,\vec{x})$ which
varies in space and time, which has its volume density be
$\rho(t,\vec{x})$. The flow of this quantity is described the the
\define{Flux vector} $\vec{j}(t,\vec{x})$. By the exact same arguments
as given with mass, we can obtain the differential form of the
conservation of $q$
\begin{equation*}
\partial_{t}\rho(t,\vec{x})-\divergence\vec{j}(t,\vec{x})=f(t,\vec{x})
\end{equation*}
where $f(t,\vec{x})$ are any external sources or sinks of $q$.
\end{node}
\end{node}
