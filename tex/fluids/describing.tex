% 25
\section{Describing Fluids}

\begin{node}\label{fluids:describing-0000}%
Consider some fluid, which has its initial position in some region, say
$\mathcal{B}_{0}\subset\RR^{3}$ (for some ``body'' at time $t_{0}$). We
take $\mathcal{B}_{0}$ to be an open subset, and we may also generalize
the number of spatial dimensions to $n>0$.
\end{node}

\begin{node}\label{fluids:describing-0001}%
As the fluid moves, its constituent particles also move, occupying a
(potentially different) region $\mathcal{B}_{t}$ at time $t$. We could
introduce a \define{Time Evolution Operator}
\begin{equation*}
\mathcal{M}_{t}\colon\mathcal{B}_{0}\to\mathcal{B}_{t}
\end{equation*}
which is parametrized by some time $t$ and transforms the initial body
to the body at the specified time.
\end{node}

\begin{node}[Lagrangian coordinates]\label{fluids:describing-0002}%
Suppose $\vec{a}$ is a point in the initial body $\vec{a}\in\mathcal{B}_{0}$.
We can parametrize the position of a fluid particle at time $t$ by $\vec{a}$,
\begin{equation*}
\vec{x}=\vec{X}(\vec{a},t)
\end{equation*}
which is such that $\vec{a}=\vec{X}(\vec{a},t_{0})$. This is the
Lagrangian coordinates for the fluid particle identified by initial
position $\vec{a}$. That is to say, $\vec{X}(\vec{a},t)$ is the position
at time $t$ of a \emph{particular fluid parcel} described by the
parameter $\vec{a}$.

\begin{node}[Parametrization]\label{fluids:describing-0003}%
We could actually choose any arbitrary parametrization of the
particle. That is to say, we don't need $\vec{a}$ to be the initial
position of the fluid. We could have chosen $\vec{a}$ to be the position
at some other particular moment, or we could choose a completely
different scheme (provided the labeling was
unique). Bennett~\cite[pg.5]{bennett2006lagrangian} points out, we could
even use thermodynamic properties at some initial time to identify fluid
parcels.

But in practice, it's usually the case that $\vec{a}$ is the initial
position of a fluid particle. Take care when reading the literature,
because someone may have a brilliant new idea which requires a
particularly idiosyncratic choice for $\vec{a}$.
\end{node} % Parametrization

\begin{node}[In Hamiltonian fluid mechanics]\label{fluids:describing-0004}%
The Lagrangian picture resembles that in analytical mechanics: we care
about the history of each particle. (It's also coincidentally the
starting point for the Hamiltonian/``Canonical'' formalism in fluid
dynamics, just as it's the starting point for the Hamiltonian formalism
in analytical mechanics.) Although mathematically and physically
intuitive, elegant, and appealing\dots it has the disadvantage that, when
we do measurements in the lab, it's at a specific point in time (not
over the history of the particle). 
\end{node} % Hamiltonian fluid mechanics
\end{node} % Lagrangian coordinates

\begin{node}[Eulerian description]\label{fluids:describing-0005}%
We could take the opposite point of view, starting with a fixed point
$\vec{x}$ inside the fluid $\vec{x}\in\mathcal{B}_{t}$. Then we take the
quantities of interest as functions $f(\vec{x},t)$. This is the
\emph{Eulerian description} of the fluid. Convention compels us to then
examine the velocity vector field $\vec{u}(\vec{x},t)$ at each point in
the fluid\dots although there's nothing to stop us from extending
$\vec{u}$ to $\RR^{n}$ in general, provided we demand $\vec{u}=\vec{0}$
outside the fluid.
\end{node} % Eulerian description

\begin{node}[Relating the two pictures]\label{fluids:describing-0006}%
The velocity vector field relates to the Lagrangian description by
taking the time derivative of the Lagrange coordinates:
\begin{equation}
\left.\frac{\partial\vec{X}(\vec{a},t)}{\partial t}\right|_{\vec{a}}=\vec{u}(\vec{X}(\vec{a},t),t).
\end{equation}
This gives us a system of first-order ordinary differential equations
we'd need to solve.


Another way to view this is that we can describe the position at time
$t$ for the fluid parcel which has initial position $\vec{a}$ at time
$t=t_{0}$ by the function $\vec{X}(\vec{a},t)$. Its time-evolution may
be described by the function
\begin{equation}
\vec{X}(\vec{a},t)=\phi_{t}(\vec{a})
\end{equation}
such that (locally) $\phi_{t}$ is invertible. That is to say, we can
write the Lagrange parameter as a function of position, simply by
``tracing backwards in time'' the trajectory the parcel at $\vec{X}$
took until we arrive at its position $\vec{a}$ at time $t_{0}$,
\begin{equation}
\vec{a}(\vec{X},t) = \phi_{t}^{-1}\bigl(\vec{X}(\vec{a},t)\bigr).
\end{equation}
Now we may introduce the \emph{fluid velocity} in the Lagrangian
description as
\begin{equation}
\vec{v}_{\text{L}}=\frac{\partial\vec{X}(\vec{a},t)}{\partial t}.
\end{equation}
Then, using these equations, the Eulerian picture may be obtained by writing:
\begin{equation}
\vec{u}(\vec{x},t) = \vec{v}_{\text{L}}(\vec{a}(\vec{x},t),t)
=\left.\frac{\partial\vec{X}(\vec{a},t)}{\partial t}\right|_{\vec{a}=\phi_{t}^{-1}(\vec{x})}.
\end{equation}


\begin{node}[Caution with notation]\label{fluids:describing-0007}%
Here I must stress the warning which Sussman and Wisdom give in
\textit{The Structure and Interpretation of Classical Mechanics}~\cite{sicm}
about ambiguity with notation. Some authors (especially from physics)
write the partial derivative with respect to, say, time to mean:
\begin{equation}
  \frac{\partial}{\partial t}f(\vec{x}(t), t)
  = \left.\frac{\partial f(\vec{q}, t)}{\partial t}\right|_{\vec{q}=\vec{x}(t)}.
\end{equation}
Undergraduates and mathematicians are understandably confounded by this
choice of notation (what ever happened to the ``chain rule''?). Well,
that's handled by the total derivative with respect to time:
\begin{equation}
\frac{\D}{\D t}f(\vec{x}(t), t)
= \left.\frac{\partial f(\vec{q}, t)}{\partial t}
  + \sum_{j}\frac{\partial f(\vec{q},t)}{\partial q_{j}}\frac{\D x_{j}(t)}{\D t}\right|_{\vec{q}=\vec{x}(t)}.
\end{equation}
This notation has been ``grandfathered in'', and there's nothing we can
do to change it. I'm so, so sorry.
\end{node} % Caution with Notation
\begin{node}\label{fluids:describing-0008}%
Also note, if we wanted to be ``completely general'', we would need to
consider derivatives with respect to the Lagrange parameter
$\vec{a}$---the curious reader may consult with pleasure Lamb's \textit{Hydrodynamics}~\cite[\S\S13--14]{lamb1945hydrodynamics}
\end{node}
\end{node} % Relating the two pictures

\begin{example}[{Acheson~\cite{acheson1990elementary}}]
Consider a 2-dimensional fluid with
\begin{equation*}
\vec{u}=(u(x,y,t), v(x,y,t))=(-\Omega y,\Omega x)
\end{equation*}
where $\Omega>0$ is a constant angular speed. We have two differential
equations
\begin{subequations}
\begin{equation}
\partial_{t}x = -\Omega y(t),\qquad x(0)=a,
\end{equation}
and
\begin{equation}
\partial_{t}y = \Omega x(t),\qquad y(0)=b.
\end{equation}
\end{subequations}
Then we see (by differentiating with respect to time again, in both
equations) that 
\begin{equation*}
\partial_{t}^{2}x = -\Omega^{2}x,\quad\mbox{and}\quad\partial_{t}^{2}y=-\Omega^{2}y
\end{equation*}
which have general solutions of the form
\begin{equation}
(x(t),y(t)) = (a\cos(\Omega t)-b\sin(\Omega t), b\cos(\Omega t) + a\sin(\Omega t)).
\end{equation}
\end{example}

\begin{example}[{Childress~\cite{childress2009introduction}}]
Consider \define{Steady Flow} in 2-dimensions. We have
\begin{equation*}
\vec{u}=(u(x,y,t), v(x,y,t))=(x,-y).
\end{equation*}
Using the Lagrangian description, we'd have
\begin{subequations}
\begin{equation}
\partial_{t}x = x,\qquad x(0)=a
\end{equation}
and
\begin{equation}
\partial_{t}y=-y,\qquad y(0)=b.
\end{equation}
\end{subequations}
This has its general solution be
\begin{equation}
\vec{X}(\vec{a},t)=(a\E^{t},b\E^{-t}).
\end{equation}
\end{example}

\begin{example}[{Acheson~\cite{acheson1990elementary}}]
Consider now the Rankine Vortex in two dimensions. In polar coordinates,
it's defined as  
\begin{equation}
\dot{\theta} = \begin{cases} \Omega & r\leq R\\
\Omega(R/r)^{2} & r\geq R
\end{cases}
\end{equation}
and $\dot{r}=0$, where $\Omega>0$ is some fixed angular speed, $R>0$ is
the width of the vortex. We see then that $r=r_{0}$ is constant, and
\begin{equation}
\theta(t) = \theta_{0} + \begin{cases} \Omega t & r\leq R\\
\Omega t (R/r)^{2} & r\geq R
\end{cases}
\end{equation}
where $\theta_{0}$ is the initial angle at time zero. We can change
coordinates to see
\begin{equation}
x(t) = r\cos(\theta(t)),\quad y(t)=r\sin(\theta(t)).
\end{equation}
This lets us determine $r$ and $\theta_{0}$ in terms of $x(0)=a$ and
$y(0)=b$. So $r^{2}=a^{2}+b^{2}$ and $\theta_{0}=\atan2(b,a)$ (recall
\zref{calculus-0017} the definition of $\atan2$). Also
note there is a singularity when $a=b=0$ since $\theta_{0}$ is not
well-defined. 
\end{example}

\begin{node}[Parcel, Particles, and the Continuum]\label{fluids:describing-0009}%
Historically, fluid mechanics really began in earnest with Euler and
Lagrange's works on fluids in the 1740s. This is where the equations of
motion for fluids were first formalized using the tools of calculus. But
it predates the atomic theory, so the intuitive picture was a literal
continuum of point particles.

The intuition seems to jump around from a continuum of point-particles,
to working with densities (an ``infinitesimal region'' of the fluid,
called a ``fluid parcel''), rather freely.

We could meaningfully ask about the validity of describing fluids as a
continuum of point-particles.

\begin{node}[Fluid parcels]\label{fluids:describing-000O}%
When Euler and friends began describing fluids in the 1700s, they
imagined a literal continuum of point-particles as their ``microscopic
molecular mental model'' for a fluid. This was interchangeable with a
fluid parcel, which is [usually] taken to be an ``infinitesimal volume''
of the fluid. As Batchelor~\cite[pg.71]{batchelor1967introduction}
notes, the position of a fluid parcel is its center-of-mass position
(which makes some crude sense to me, drawing in analogy to Euler's laws
of motion for a rigid body --- this is wrong, but appealing).

In practice, the fluid parcel --- this infinitesimal volume of fluid ---
describes the fundamental constituent of a fluid. This suggests thinking
of a fluid parcel as a Parallelepiped. Its volume can change (this is
the determinant of the Jacobian matrix
$\D^{3}x=\det(J)\,\D^{3}a=\det(\partial\vec{x}/\partial\vec{a})\,\D^{3}a$,
after all) as can its density $\rho$. But it is indivisible, we cannot
``split'' one parcel into two or more.

This topic is usually quickly brushed over, discussion omitted from any
real consideration. Pozrikidis~\cite[\S1.4]{pozrikidis2009fluid}
discusses fluid parcels from a rigorous statistical mechanical approach,
and why the continuum hypothesis is needed for a well-defined limit of
the volume $V$ of a finite fluid parcel going to an infinitesimal volume
element. After all, the velocity for a finite volume element would be
the mean value of velocities for its constituent particles
\begin{equation}
\langle\vec{u}\rangle=\frac{1}{N}\sum^{N}_{j=1}\vec{u}_{j},
\end{equation}
which tends to be ill-defined when working in a spheric
$V=\frac{4}{3}\pi\varepsilon^{3}$ in the $\varepsilon\to0$ limit.
The continuum hypothesis prescribes: as the volume of a fluid parcel
tends to zero, the limit of the average molecular velocity is computed
before the discrete nature of the fluid becomes apparent.

\begin{puzzle}[Fluids described by tensor densities]
If we take seriously the mental model proposed by fluid parcels, then we
should think of $\vec{u}$ as a vector density (i.e., a rank-1 tensor density). 
But it doesn't: it transforms like a vector (i.e., a rank-1 tensor).
What's going on here? Is there a sensible formulation of the idealized
mental model for fluids using tensor densities? 
\end{puzzle}

\begin{puzzle}[Molecular description of fluids, Boltzmann equation]
It is equally clear that this mental model of fluid parcels is a very
distinct idealization than what we normally think of (with molecules).
As I understand it, the attempt to describe fluids using a more
molecular approach is precisely the Boltzmann equation. How applicable
is this to studying fluids?
\end{puzzle}
\end{node} % Fluid parcels

\begin{node}[Continuum Hypothesis]\label{fluids:describing-000A}%
A fluid parcel may be described like a point-particle, and has the
physical characteristics of the bulk (e.g., velocity, pressure, density,
etc.). In particular, any small finite volume of a fluid contains a
great number of molecules.

\begin{node}[Remark]\label{fluids:describing-000B}%
I have seen the literature argue the ``amount of matter'' in a fluid
parcel is constant (which means its volume can vary, otherwise density
is constant). I'm not sure if this is standard or not? If so, I suppose
we could conceptually think of the number of molecules in a parcel is
fixed\dots but this may be ``chalk and cheese''.
\end{node} % Remark

\begin{node}[Empirical considerations]\label{fluids:describing-000H}%
Howard Brenner~\cite{brenner2005kinematics,brenner2005navier,brenner2006fluid} has proposed a modification to this classical picture of
a fluid as a continuum of point particles. The basic alternative has two
different notions of velocity, a ``kinematical'' velocity
$\vec{u}_{\text{kin}}$ used for the conservation of mass, and a
``dynamical'' velocity $\vec{u}_{\text{dyn}}$ used in the equations
of motion. The two are related by some dispersion relation like
$\vec{u}_{\text{dyn}}-\vec{u}_{\text{kin}}=K\grad\ln(p)$ where $K$ is a
constant that depends on the fluid. This turns out to have nicer
mathematical properties; see, e.g., Feireisl and Vasseur~\cite{feireisl2010new}.
Further, Greenshields and Reese~\cite{greenshields2007structure} have
shown that Brenner's model gives good predictions for shock waves in
argon in the range Mach $1.0$--$12.0$ (whereas convention Navier--Stokes
equations fail beyond Mach $2.0$).
\end{node} % Empirical considerations
\end{node} % Continuum Hypothesis

\begin{definition}\label{fluids:describing-000C}%
The \define{Knudsen number} for a body is
\begin{equation}
\Kn := \lambda/L
\end{equation}
where $L$ is the length scale of the body, and $\lambda$ is the mean
free path of its constituent particles. Observe this is a dimensionless
quantity, and a positive real number.

\begin{example}[Knudsen number of air]\label{fluids:describing-000E}%
Air at 1013 hPa has a mean free path of approximately 68 nm (see
Jennings~\cite{jennings1988air}). For geophysical models, the length
scale would be on the order of the scale atmospere height $8.5$
kilometers. Therefore the knudsen number for air would be
\begin{equation}
\Kn\approx\frac{68\times 10^{-9}}{8.5\times 10^{3}}\approx8\times 10^{-12}\sim10^{-11}.
\end{equation}
Hence fluid mechanics would be a good approximation at this scale.
\end{example}

\begin{example}[Cup of water]\label{fluids:describing-000G}%
Water can be estimated to have a mean free path of
$\lambda\approx2.5\times10^{-10}~\mathrm{m}$. For a cup of water, which
holds 1 cup (approximately 236588 cubic millimeters), its scale would be
the cuberoot of this number $L\approx 62\times10^{-3}~\mathrm{m}$.
Therefore, the Knudsen number for a cup of water is approximately
\begin{equation}
\Kn\approx\frac{2.5\times10^{-10}}{62\times10^{-3}}\approx4\times10^{-9}.
\end{equation}
Fluid mechanics describes a cup of water very well, as we would expect.
\end{example}
\end{definition}

\begin{node}[Applicability of fluid mechanics]\label{fluids:describing-000D}%
The heuristic is: we may apply fluid mechanics if $\Kn\ll1$. If
$\Kn<0.01$, there's no problem. If $0.01<\Kn<0.1$, then it's
doable but slipping occurs. For $0.1<\Kn<10$ things gets dicey. The
cutoff is rather fuzzy, but larger values require resorting to
statistical mechanics. In the cases we normally think of with fluid
mechanics, we find $\Kn\ll0.01$ (as demonstrated by the examples above).
\end{node} % Applicability of fluid mechanics

\begin{node}[Hill--Mandel principle]\label{fluids:describing-000F}%
More generally, in the field of continuum mechanics (which describes
fluids \emph{and} elastic bodies), the criterion used for determining
the appropriateness of applicability is the Hill--Mandel principle. If
$F$ is the deformation gradient and $P$ is the stress tensor in
equilibrium, then
\begin{equation}
\langle P : F\rangle = \langle P \rangle : \langle F \rangle
\end{equation}
where the brackets indicate microscopic averaging, and
\begin{equation*}
A : B = \tr(A\transpose{B}) = \sum_{i,j}(A)_{ij}(B)_{ij}.
\end{equation*}
For more about the Hill--Mandel condition, see, e.g., Liu and
Reina~\cite{liu2016discrete}. I have not, however, seen any discussion
relating the Hill--Mandel condition to the continuum hypothesis for
fluid mechanics.
\end{node} % Hill--Mandel principle
\end{node} % Continuum Hypothesis

\begin{node}[Kinematics]\label{fluids:describing-000I}%
Kinematics means we are assuming knowledge of fluid motion, i.e., we are
given an explicit Eulerian velocity field $\vec{u}(\vec{x},t)$ or we are
given explicit Lagrangian coordinates $\vec{x}=\vec{X}(\vec{a},t)$.

\begin{node}[Pathline]\label{fluids:describing-000K}%
The pathline for a fluid element is simply the path of that element as a
function of time. These are simply $\vec{x}$ obtained either by fiat (if
we are working in the Lagrangian coordinates) or by solving a system of
ordinary differential equations
\begin{equation*}
\frac{\D\vec{x}(t)}{\D t}=\vec{u}(\vec{x}(t),t)
\end{equation*}
if working in the Eulerian coordinates.
\end{node} % pathline

\begin{node}[Streamlines]\label{fluids:describing-000J}%
A kinematical characterization of fluids are through streamlines, which
are the particle paths for steady flow. In unsteady flow, we ned to
consider \emph{instantaneous} steamlines (at a specific instant of time
$t$).
In three dimensions the instantaneous streamlines are the orbits of
$\vec{u}(\vec{x},t)=\bigl(u(x,y,z,t),v(x,y,z,t),w(x,y,z,t)\bigr)$
at time $t$. These are the integral curves satisfying
\begin{equation*}
\frac{\D x}{u}=\frac{\D y}{v}=\frac{\D z}{w}.
\end{equation*}
This is shorthand for stating
\begin{equation}
(w\,\D y-v\,\D z)\widehat{\vec{x}}+(u\,\D z-w\,\D x)\widehat{\vec{y}}
+(v\,\D x-u\,\D y)\widehat{\vec{z}}=0,
\end{equation}
which has each component set to zero, giving rise to three differential
equations defining the streamline:
\begin{subequations}
\begin{align}
\frac{\D y}{\D z} & =\frac{v}{w} \\
\frac{\D z}{\D x} & =\frac{w}{u} \\
\frac{\D y}{\D x} & =\frac{v}{u}.
\end{align}
\end{subequations}
\end{node} % streamline

\begin{node}[Streakline]\label{fluids:describing-000L}%
A streakline associated with a particular point $P$ in space which has
the fluid moving past it. All points which pass through $P$ are said to
form the streakline of point $P$. (Smoke emitted from a chimney is an
example of a streakline.)

Childress~\cite[\S1.1.1]{childress2009introduction} introduces a notion
of \emph{generalized Lagrangian coordinates} $\vec{x}=\vec{X}(\vec{a},t,t_{\vec{a}})$
which is the position at time $t$ of the particle which was located at $\vec{a}$
at time $t_{\vec{a}}$ (when $t_{\vec{a}}=t_{0}$, this is just the usual
Lagrangian coordinates). A streakline observed at time $t>0$ which was
started at time $t=0$ is given by
$\vec{x}=\vec{X}(\vec{a},t,t_{\vec{a}})$ for $0<t_{\vec{a}}<t$.
\end{node}

\begin{example}[{Childress~\cite[1.4]{childress2009introduction}}]\label{fluids:describing-000M}%
Consider the velocity vector field $(u,v)=(y,-x+\varepsilon\cos(\omega t))$.
What are the streamlines? What are the streaklines?

We know the streamlines satisfy
\begin{equation}
\frac{\D x}{y}=\frac{\D y}{-x+\varepsilon\cos(\omega t)}\implies\frac{-x+\varepsilon\cos(\omega t)}{y(x)}=\frac{\D y(x)}{\D x}.
\end{equation}
This has a solution
\begin{equation}
\bigl(x-\varepsilon\cos(\omega t)\bigr)^{2} + y^{2}=\mbox{constant}.
\end{equation}
Observe this is not a constant with respect to time.

The streaklines require moving from the first-order differential
equations
\begin{align*}
\frac{\D x}{\D t} &= y(t)\\
\frac{\D y}{\D t} &= -x(t)+\varepsilon\cos(\omega t)
\end{align*}
to a second-order system of differential equations
\begin{align*}
\frac{\D^{2} x}{\D t^{2}} &= -x(t)+\varepsilon\cos(\omega t)\\
\frac{\D^{2} y}{\D t^{2}} &= -y(t)-\varepsilon\omega\sin(\omega t)
\end{align*}
This has the general solution
\begin{subequations}
\begin{align}
x(t) &= -\frac{\varepsilon}{\omega^{2}-1}\cos(\omega t) + C_{1}\cos(t)+C_{2}\sin(t)\\
y(t) &= \frac{\varepsilon\omega}{\omega^{2}-1}\sin(\omega t)+C_{3}\cos(t)+C_{4}\sin(t),
\end{align}
where we use the initial condition $x(t_{a})=a$ and $y(t_{a})=b$, as
well as $x'(t_{a})=b$ and $y'(t_{a})=-a+\varepsilon\cos(\omega t_{a})$:
\begin{align}
a &= -\frac{\varepsilon}{\omega^{2}-1}\cos(\omega t_{a}) + C_{1}\cos(t_{a})+C_{2}\sin(t_{a})\\
b &= \frac{\varepsilon\omega}{\omega^{2}-1}\sin(\omega t_{a})+C_{3}\cos(t_{a})+C_{4}\sin(t_{a})\\
b &= C_{2}\cos(t_{a})-C_{1}\sin(t_{a})-\frac{\varepsilon\omega}{1-\omega^{2}}\sin(t_{a}\omega)\\
-a+\varepsilon\cos(\omega t_{a}) &= C_{4}\cos(t_{a})-C_{3}\sin(t_{a})
-\frac{\varepsilon\omega^{2}}{1-\omega^{2}}\cos(t_{a}\omega)
\end{align}
\end{subequations}
We see that $C_{2}=C_{3}=:B$ and $C_{1}=-C_{4}=:A$, so we can solve for these
and find
\begin{subequations}
\begin{align}
x(t) &= -\frac{\varepsilon}{\omega^{2}-1}\cos(\omega t) + A\cos(t)+B\sin(t)\\
y(t) &= \frac{\varepsilon\omega}{\omega^{2}-1}\sin(\omega t)+B\cos(t)-A\sin(t)
\end{align}
where (and Childress errs in Eq (1.6) of their book! Childress's value
of $B$ is wrong, note the signs of the terms involving $\varepsilon$!):
\begin{align*}
A &= -b\sin(t_{a})+\frac{\varepsilon\omega}{\omega^{2}-1}\sin(t_{a})\sin(t_{a}\omega)+a\cos(t_{a}) \frac{\varepsilon}{\omega^{2}-1}\cos(t_{a})\cos(t_{a}\omega)\\
B &= a\sin(t_{a}) + b\cos(t_{a}) +\frac{\varepsilon}{\omega^{2}-1}\cos(t_{a}\omega)\sin(t_{a})-\frac{\varepsilon\omega}{\omega^{2}-1}\cos(t_{a})\sin(t_{a}\omega)
\end{align*}
\end{subequations}
We see that upon evaluation (and a lot of simplification) $x(t_{a})=a$,
$y(t_{a})=b$, $x'(t_{a})=b$, and
$y'(t_{a})=-a+\varepsilon\cos(t_{a}\omega)$, as desired.
\end{example}
\end{node} % kinematics

\begin{node}[{On ``Ideal'' fluids}]\label{fluids:describing-000N}%
In thermodynamics, we have a notion of an ``ideal gas''. There is no
exact analog in fluid mechanics. Often times we will have an
incompressible fluid with zero viscosity, which is a suitably nice
mathematical abstraction.

Recall, viscosity is a sort of ``friction'' which fluid parcels would
experience when moving against each other.

But different authors have different notions of ``ideal fluids''.  For
example, Landau and Lifshitz~\cite[\S2]{landau1987fluids}, ``fluids in
which thermal conductivity and viscosity are unimportant are said to be
\emph{ideal}.''
\end{node}