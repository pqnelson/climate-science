% 6
\section{Euler Fluids}

\begin{node}[Material Derivative]\label{fluids:euler-flow-0000}%
Suppose we have some physical quantity $f$ of the fluid which can be
``attached'' to a fluid parcel (e.g., temperature or pressure or density).
For the sake of discussion, suppose as the fluid parcel moves, this
property is invariant in time (``conserved''). We can express this by
the equation
\begin{equation}
\left.\frac{\partial f}{\partial t}\right|_{\vec{a}}=0,
\end{equation}
since this means the time derivative is taken with the ``particle label''
$\vec{a}$ fixed (i.e., is taken as we move with the fluid parcel).

How do we express this in the Eulerian description? Well, we select a
point $\vec{x}$ and seek to express the invariance in terms of
properties of the fluid \emph{at this point}. Since the fluid is
generally moving at the point, we need to bring in velocity. The way we
do this is to differentiate $f(\vec{x}(\vec{a},t),t)$ using the chain
rule and recalling how to relate Euler and Lagrange pictures \zref{fluids:describing-0006},
\begin{calculation}
  \left.\frac{\partial f(\phi_{t}(\vec{a}),t)}{\partial t}\right|_{\vec{a}}
\step{using the chain rule with fixed $\vec{a}$}
  \frac{\partial f(\vec{x},t)}{\partial t} +
  \left.\frac{\partial\phi_{t}(\vec{a})}{\partial t}
  \frac{\partial f(\phi_{t}(\vec{a}),t)}{\partial t}\right|_{\vec{a}}
\step{rewriting using Euler description}
  \partial_{t}f(\vec{x},t)+(\vec{u}(\vec{x},t)\cdot\grad) f(\vec{x},t).
\end{calculation}
which must vanish (since it's conserved). Stokes introduced the notation
\begin{equation}
\frac{\materialD}{\materialD t}=\partial_{t}+\vec{u}\cdot\grad
\end{equation}
for the \define{Material Derivative} with respect to the fluid. This
describes the flow of some quantity with respect to time in the Eulerian
picture. 
\end{node}

\begin{node}[Incompressible fluids]\label{fluids:euler-flow-0003}%
When the fluid parcel's volume is preserved (i.e., constant with respect
to time), then we say the fluid is \define{Incompressible}. This happens
most often with liquids. We can express this condition by writing the
Jacobian for the fluid parcel labeled $\vec{a}$,
\begin{equation}
J_{ij}=\left.\frac{\partial x_{i}}{\partial a_{j}}\right|_{t},
\end{equation}
and observing that when $\D a_{1}\cdots\D a_{N}$ is the volume of the
fluid parcel at time $t=t_{0}$, then $\det(J)\D a_{1}\cdots\D a_{N}$ is
the volume of that parcel at time $t$. For these two volumes to be
equal, this forces the condition
\begin{equation}
\det(J)=1.
\end{equation}

\begin{theorem}\label{fluids:euler-flow-0004}%
A fluid is incompressible if and only if $\divergence\vec{u}=0$.
\end{theorem}
\begin{proof}
We can express incompressibility using the material derivative as
\begin{equation}
\frac{\materialD}{\materialD t}\det(J)=0.
\end{equation}
We can prove by induction on the number of spatial dimensions $n$ that
\begin{equation}
\frac{\materialD}{\materialD t}\det(J)=(\divergence\vec{u})\det{J}.
\end{equation}
But this means either $\det(J)=0$ or $\divergence\vec{u}=0$. Since
$\det{J}\neq0$, this gives us the result.
\end{proof}
\end{node}


\begin{node}[Continuity equation]\label{fluids:euler-flow-0002}%
We see that mass conservation gives us the equation
\begin{subequations}
\begin{equation}
\partial_{t}\rho+\divergence(\rho\vec{u})=0.
\end{equation}
This is the \define{Continuity Equation}. We can rewrite it as
\begin{equation}
\frac{\materialD\rho}{\materialD t}+\rho\divergence\vec{u}=0.
\end{equation}
\end{subequations}
For incompressible fluids, Theorem~\ref{fluids:euler-flow-0004} gives us 
$\materialD\rho/\materialD t=0$.
\end{node}

\begin{node}[Equations of motion]\label{fluids:euler-flow-0001}%
We can recall that Newton's equations of motion are precisely
$\D(m\vec{v})/\D t = \vec{F}$. Although it is tempting to just replace
mass $m$ by density $\rho$, what we should do is consider the integral
over the volume of the fluid parcel
\begin{equation}
\frac{\D}{\D t}\int_{\mathcal{B}_{t}}\rho\vec{u}\,\D^{n}x = \vec{F}.
\end{equation}
Then we can expand the left-hand side more generally as
\begin{equation}
\frac{\D}{\D t}\int_{\mathcal{B}_{t}}\rho f\,\D^{n}x = 
\int_{\mathcal{B}_{t}}\left[\partial_{t}(\rho f)+\divergence(\rho f\vec{u})\right]\D^{n}x
\end{equation}
We see that $\divergence(\rho
f\vec{u})=f\divergence(\rho\vec{u})+\rho\vec{u}\cdot\grad f$, so the
integrand splits into a part which vanishes by the conservation of mass,
and the produce of density with the material derivative of $f$
\begin{equation}
\frac{\D}{\D t}\int_{\mathcal{B}_{t}}\rho f\,\D^{n}x = 
\int_{\mathcal{B}_{t}}\rho\frac{\materialD f}{\materialD t}\D^{n}x.
\end{equation}
Then we return to our original equation, replacing $f$ with the
components of the velocity vector $u_{j}\widehat{\vec{e}}_{j}$
and sum over the components to obtain
\begin{equation}
\int_{\mathcal{B}_{t}}\rho\frac{\materialD\vec{u}}{\materialD t}\D^{n}x = \vec{F}.
\end{equation}
Then we have obtained the Euler equations of motion, perhaps better
named the conservation of mass,
\begin{equation}
\boxed{\rho\frac{\materialD\vec{u}}{\materialD t}=\vec{f}}
\end{equation}
where $\vec{f}$ is the force acting on the fluid, per unit volume.

\begin{node}[Pressure gradient force]\label{fluids:euler-flow-0005}%
In particular, all fluids will have the pressure gradient force as part
of the force density. This is literally $\nabla p$. Really, we are
sweeping subtleties under the rug here, because we should write the
force $\vec{F}$ as
\begin{equation*}
\vec{F}=\int_{\mathcal{B}_{t}}\vec{f}_{\text{body}}\,\D^{n}x+\oint_{\partial\mathcal{B}_{t}}\sigma\cdot\mathcal{N}\,\D\Sigma
\end{equation*}
where $\sigma$ is the stress tensor, and describes surface force
interactions. For most fluids, $\sigma=\diag(-p,-p,-p)$ where $p$ is pressure.
\end{node}
\end{node}