% 23
\section{Euler Flow Equations}

\begin{node}[Material Derivative]\label{fluids:euler-flow-0000}%
Suppose we have some physical quantity $f$ of the fluid which can be
``attached'' to a fluid parcel (e.g., temperature or pressure or density).
For the sake of discussion, suppose as the fluid parcel moves, this
property is invariant in time (``conserved''). We can express this by
the equation
\begin{equation}
\left.\frac{\partial f}{\partial t}\right|_{\vec{a}}=0,
\end{equation}
since this means the time derivative is taken with the ``particle label''
$\vec{a}$ fixed (i.e., is taken as we move with the fluid parcel).

How do we express this in the Eulerian description? Well, we select a
point $\vec{x}$ and seek to express the invariance in terms of
properties of the fluid \emph{at this point}. Since the fluid is
generally moving at the point, we need to bring in velocity. The way we
do this is to differentiate $f(\vec{x}(\vec{a},t),t)$ using the chain
rule and recalling how to relate Euler and Lagrange pictures \zref{fluids:describing-0006},
\begin{calculation}
  \left.\frac{\partial f(\phi_{t}(\vec{a}),t)}{\partial t}\right|_{\vec{a}}
\step{using the chain rule with fixed $\vec{a}$}
  \frac{\partial f(\vec{x},t)}{\partial t} +
  \left.\frac{\partial\phi_{t}(\vec{a})}{\partial t}
  \frac{\partial f(\phi_{t}(\vec{a}),t)}{\partial\phi_{t}(\vec{a})}\right|_{\vec{a}}
\step{rewriting using Euler description}
  \partial_{t}f(\vec{x},t)+(\vec{u}(\vec{x},t)\cdot\grad) f(\vec{x},t).
\end{calculation}
which must vanish (since it's conserved). Stokes introduced the notation
\begin{equation}
\frac{\materialD}{\materialD t}=\partial_{t}+\vec{u}\cdot\grad
\end{equation}
for the \define{Material Derivative} with respect to the fluid. This
describes the flow of some quantity with respect to time in the Eulerian
picture. 
\end{node} % Material Derivative

\begin{puzzle}[Cohomology and material derivative]
Is there some natural description of the material derivative using
differential forms? If so, what is the cohomology of the resulting
material derivative operator? Does it tell us anything useful or
interesting? 
\end{puzzle}

\begin{node}[Incompressible fluids]\label{fluids:euler-flow-0003}%
When the fluid parcel's volume is preserved (i.e., constant with respect
to time), then we say the fluid is \define{Incompressible}. This happens
most often with liquids. We can express this condition by writing the
Jacobian for the fluid parcel labeled $\vec{a}$,
\begin{equation}
J_{ij}=\left.\frac{\partial x_{i}}{\partial a_{j}}\right|_{t},
\end{equation}
and observing that when $\D a_{1}\cdots\D a_{N}$ is the volume of the
fluid parcel at time $t=t_{0}$, then $\det(J)\D a_{1}\cdots\D a_{N}$ is
the volume of that parcel at time $t$. For these two volumes to be
equal, this forces the condition
\begin{equation}
\det(J)=1.
\end{equation}

\begin{theorem}\label{fluids:euler-flow-0004}%
A fluid is incompressible if and only if $\divergence\vec{u}=0$.
\end{theorem}
\begin{proof}
We can express incompressibility using the material derivative as
\begin{equation}
\frac{\materialD}{\materialD t}\det(J)=0.
\end{equation}
We can prove by induction on the number of spatial dimensions $n$ that
\begin{equation}
\frac{\materialD}{\materialD t}\det(J)=(\divergence\vec{u})\det{J}.
\end{equation}
But this means either $\det(J)=0$ or $\divergence\vec{u}=0$. Since
$\det{J}\neq0$, this gives us the result.
\end{proof}
\end{node} % Incompressible fluids


\begin{node}[Continuity equation]\label{fluids:euler-flow-0002}%
We see that mass conservation gives us the equation
\begin{subequations}
\begin{equation}
\partial_{t}\rho+\divergence(\rho\vec{u})=0.
\end{equation}
This is the \define{Continuity Equation}. We can rewrite it as
\begin{equation}
\frac{\materialD\rho}{\materialD t}+\rho\divergence\vec{u}=0.
\end{equation}
\end{subequations}
For incompressible fluids, Theorem~\ref{fluids:euler-flow-0004} gives us 
$\materialD\rho/\materialD t=0$.
\end{node} % Continuity equation

\begin{node}[Equations of motion]\label{fluids:euler-flow-0001}%
We can recall that Newton's equations of motion are precisely
$\D(m\vec{v})/\D t = \vec{F}$. Although it is tempting to just replace
mass $m$ by density $\rho$, what we should do is consider the integral
over the volume of the fluid parcel
\begin{equation}
\frac{\D}{\D t}\int_{\mathcal{B}_{t}}\rho\vec{u}\,\D^{n}x = \vec{F}.
\end{equation}
Then we can expand the left-hand side more generally as
\begin{equation}
\frac{\D}{\D t}\int_{\mathcal{B}_{t}}\rho f\,\D^{n}x = 
\int_{\mathcal{B}_{t}}\left[\partial_{t}(\rho f)+\divergence(\rho f\vec{u})\right]\D^{n}x
\end{equation}
We see that $\divergence(\rho
f\vec{u})=f\divergence(\rho\vec{u})+\rho\vec{u}\cdot\grad f$, so the
integrand splits into a part which vanishes by the conservation of mass,
and the produce of density with the material derivative of $f$
\begin{equation}
\frac{\D}{\D t}\int_{\mathcal{B}_{t}}\rho f\,\D^{n}x = 
\int_{\mathcal{B}_{t}}\rho\frac{\materialD f}{\materialD t}\D^{n}x.
\end{equation}
Then we return to our original equation, replacing $f$ with the
components of the velocity vector $u_{j}\widehat{\vec{e}}_{j}$
and sum over the components to obtain
\begin{equation}
\int_{\mathcal{B}_{t}}\rho\frac{\materialD\vec{u}}{\materialD t}\D^{n}x = \vec{F}.
\end{equation}
Then we have obtained the Euler equations of motion, perhaps better
named the conservation of mass,
\begin{equation}
\boxed{\rho\frac{\materialD\vec{u}}{\materialD t}=\vec{f}}
\end{equation}
where $\vec{f}$ is the force acting on the fluid, per unit volume.

\begin{node}[Pressure gradient force]\label{fluids:euler-flow-0005}%
In particular, all fluids will have the pressure gradient force as part
of the force density. This is literally $\nabla p$. Really, we are
sweeping subtleties under the rug here, because we should write the
force $\vec{F}$ as
\begin{equation*}
\vec{F}=\int_{\mathcal{B}_{t}}\vec{f}_{\text{body}}\,\D^{n}x+\oint_{\partial\mathcal{B}_{t}}\sigma\cdot\unitnormal\,\D\Sigma
\end{equation*}
where $\sigma$ is the stress tensor, and describes surface force
interactions. Making any statement about $\sigma$ requires thinking
about the  For an inviscous fluid (i.e., fluids obeying the Euler
flow equations), $\sigma=\diag(-p,-p,-p)$ where $p$ is [static] pressure.
In this case, the Euler flow equations become
\begin{equation}
\rho\frac{\materialD\vec{u}}{\materialD t}=\vec{f}-\grad p,
\end{equation}
where $\vec{f}$ are the body forces acting on the fluid.
\end{node} % Pressure gradient force

\begin{node}[Lagrangian description]\label{fluids:euler-flow-0006}%
Using Lagrangian description of fluids, we would have the Euler
equations become
\begin{equation}
\left.\frac{\partial^{2}\vec{X}(\vec{a},t)}{\partial t^{2}}\right|_{\vec{a}}
=\frac{-1}{\rho}J^{-1}\grad_{\vec{a}}p+\vec{f},
\end{equation}
where $J^{-1}$ is the inverse of the Jacobian matrix,
$\grad_{\vec{x}}p=J^{-1}\grad_{\vec{a}}p$, and $\vec{f}$ are the body
forces acting on the fluid.
\end{node} % Lagrangian description
\end{node} % Equations of motion

\begin{node}[Hydrostatics]\label{fluids:euler-flow-0007}%
Hydrostatics is concerned with fluids at rest (specifically $\vec{u}=\vec{0}$),
usually in the presence of gravity. Euler's equations of motion then
have the left-hand side vanish, so
\begin{equation}
\vec{0}=\frac{-1}{\rho}\grad p + \vec{g}\implies\grad p =\rho\vec{g}.
\end{equation}
If there is no external force (e.g., if we ``turned off'' gravity), then
we would have $\grad p=\vec{0}$ which implies $p$ is constant.

If $\vec{g}=-g\widehat{\vec{z}}$ and $\rho$ is constant, then we have
\[\partial_{x}p=\partial_{y}p=0,\qquad\partial_{z}p=-\rho g.\]
We can solve this as
\[p=-\rho gz+\mbox{constant}.\]
If the fluid at rest has a free surface at height $h$, where the
pressure is uniformly equal to $p_{0}$, then this surface must be the
horizontal plane $z=h$. From these initial conditions $p=p_{0}$ for
$z=h$, we find the constant of integration
\begin{equation}\label{eq:fluids:euler-flow:hydrostatic}
p=p_{0}+\rho g(h-z).
\end{equation}


\begin{node}[For large bodies]\label{fluids:euler-flow-0008}%
When we have large masses of a liquid (or a gas) the density $\rho$
cannot be assumed to be constant. This applies especially for gases
(e.g., the atmosphere). If we suppose the fluid is in mechanical
equilibrium and thermal equilibrium, then the temperature is the same at
every point. We may use the familiar thermodynamics relation
\begin{subequations}
\begin{equation}
\D\GibbsFreeEnergy=-s\,\D T + V\,\D p 
\end{equation}
where $\GibbsFreeEnergy$ is the Gibbs free energy function, $V$ is the specific
volume, $s$ is the entropy per unit mass. For constant temperature, this
becomes
\begin{equation}
\D\GibbsFreeEnergy=V\,\D p=\D p/\rho.
\end{equation}
Then $(\grad p)/\rho$ in this particular case is equal to $\grad\GibbsFreeEnergy$, so
the equation becomes
\begin{equation}
\grad\GibbsFreeEnergy=\vec{g}\implies\grad(\GibbsFreeEnergy+gz)=0,
\end{equation}
hence
\begin{equation}
\GibbsFreeEnergy+gz=\mbox{constant}.
\end{equation}
\end{subequations}
\end{node} % For large bodies

\begin{definition}\label{fluids:euler-flow-000L}%
We say a fluid is in \define{Mechanical Equilibrium} if it exhibits no
macroscopic motion.

I take this term from Landau and Lifshitz~\cite[\S4]{landau1987fluids},
even though they don't clearly define it anywhere in their 10-volume series.
\end{definition}

\begin{node}[Atmosphere]\label{fluids:euler-flow-0009}%
Another consequence of Eq~\eqref{eq:fluids:euler-flow:hydrostatic},
if a fluid (like the atmosphere) is in mechanical equilibrium in a
gravitational field, then the pressure in it can be a function only of
the altitude $z$. Otherwise if the pressure were different at different
points with the same altitude, motion would occur. It then follows from 
Eq~\eqref{eq:fluids:euler-flow:hydrostatic} that the density
\begin{equation}\label{eq:fluids:euler-flow:hydrostatic:atmosphere}
\rho=\frac{-1}{g}\frac{\D p}{\D z}
\end{equation}
is a function of $z$ only. The pressure and density together determine
the temperature (which is a function of $z$ only).

Using the ideal gas law, we can write
\begin{equation*}
\rho = \frac{Mp}{R\temperature}
\end{equation*}
where $M$ is the mean molecular mass of the gas, $R$ is the gas constant.
When we plug this into Eq~\eqref{eq:fluids:euler-flow:hydrostatic:atmosphere},
separation of variables gives us
\begin{equation}\label{eq:fluids:euler-flow:hydrostatic:atmosphere:diff-eq}
\frac{\D p}{p} = \frac{-\D z}{R\temperature/(Mg)}
               = \frac{-\D z}{k_{B}\temperature/(mg)}.
\end{equation}
If $\temperature$ is constant, then $H=k_{B}\temperature/(mg)$ is
constant, and the general solution looks like
\begin{equation}
p=p_{0}\exp(-z/H)
\end{equation}
where $p_{0}$ is the pressure at height $z=0$. For Earth's atmosphere
$p_{0}=\qty{1.01e5}{\pascal}$, the mean molecular mass of dry air has
$m=\qty{4.808e-26}{\kilo\gram}$. Therefore
$H/\temperature=\qty{29.26}{\meter\per\kelvin}$. 
For $\temperature=\qty{290}{\kelvin}$, we have $H=\qty{8500}{\meter}$.

\begin{node}[Linear temperature]\label{fluids:euler-flow-000B}%
When $\temperature(z)=T_{0}+T_{1}z$, we solve the differential Equation~\eqref{eq:fluids:euler-flow:hydrostatic:atmosphere:diff-eq},
\begin{equation}
p(z) = p_{0}\left(1 - \frac{z T_{1}}{T_{0}}\right)^{gm/k_{B}T_{1}}.
\end{equation}
Observe as $T_{1}\to0$, this recovers the previous case. However, if we
try to solve for the altitude $z$ for which $p(z)=p_{0}\exp(-1)$ (which
is the equivalent condition for $p(H)=p_{0}\exp(-1)$ when temperature
was constant with respect to altitude), setting $h=H/T_{0}$, we find
\begin{equation*}
\left(1 - \frac{z T_{1}}{T_{0}}\right)^{1/hT_{1}}=\exp(-1).
\end{equation*}
Solving for $z$, we end up with
\begin{equation*}
  \exp(-hT_{1})=1 - \frac{z T_{1}}{T_{0}}
  \implies
  z = (1-\E^{-hT_{1}})\frac{T_{0}}{T_{1}}.
\end{equation*}
The limit as $T_{1}\to 0$ gives us $z=H$ as expected. In fact, we can
series expand $z(T_{1})$ about $T_{1}\approx0$ to find
\begin{equation}
z\approx hT_{0}-\frac{1}{2}h^{2}T_{0}T_{1}+\frac{1}{3!}h^{3}T_{0}T_{1}^{2}-\frac{1}{24}h^{4}T_{0}T_{1}^{3}+\bigO(T_{1}^{4}).
\end{equation}
Since for Earth $T_{1}\sim-10^{-4}\unit{\kelvin\per\meter}$ and
$h\sim10^{3/2}\unit{\meter\per\kelvin}$, we'd have $hT_{1}\sim10^{-5/2}$,
so even to quadratic order in $hT_{1}$ we'd have corrections of order $10^{-5}$.
So we could keep a linear approximation in $T_{1}$ and still have a
decent approximation.

If we take $T_{0}=\qty{290}{\kelvin}$ and
$T_{1}=\qty{-9e-4}{\kelvin\per\meter}$ (which will match the temperature
at the Kalman line),
then $H_{\text{lin}}\approx\qty{8598.11}{\meter}$.
If we match the temperature at the stratopause,
$T_{1}=\qty{-4e-4}{\kelvin\per\meter}$,
then $H_{\text{lin}}\approx\qty{8535.25}{\meter}$. 
\end{node} % Linear temperature

\begin{node}[Quadratic temperature]\label{fluids:euler-flow-000C}%
When the temperature is quadratic in the altitude $\temperature(z)=T_{0}+T_{1}z+T_{2}z^{2}$,
we solve the differential Equation~\eqref{eq:fluids:euler-flow:hydrostatic:atmosphere:diff-eq}
with initial condition $p(0)=p_{0}$ for
\begin{equation}
p(z) = p_{0}\exp \left(\frac{2 g m \left(\arctan\left(\frac{T_{1}}{\sqrt{4
    T_{0} T_{2}-T_{1}^{2}}}\right)-\arctan\left(\frac{T_{1}+2 T_{2} z}{\sqrt{4 T_{0}
    T_{2}-T_{1}^{2}}}\right)\right)}{k_{B} \sqrt{4 T_{0} T_{2}-T_{1}^{2}}}\right)
\end{equation}
We suppose $T_{1}<0$ and $T_{2}>0$. Then $T'(z_{\text{cr}})=T_{1}+2z_{\text{cr}}T_{2}=0$ for $z_{\text{cr}}=\frac{-1}{2}T_{1}/T_{2}\approx\qty{10}{\kilo\meter}$.
We see $T(z_{\text{cr}})=T_{0}+\frac{1}{2}T_{1}z_{\text{cr}}\approx\qty{215}{\kelvin}$
and $T_{0}\approx\qty{285}{\kelvin}$ implies $T_{1}\approx\qty{-1.4e-2}{\kelvin\per\meter}$
and therefore $T_{2}=\qty{7e-7}{\kelvin\per\meter\squared}$. Using these
values, solving $p(H_{\text{quad}})=p_{0}\exp(-1)$ for $H_{\text{quad}}$
gives us for Earth
\begin{equation}
  \begin{split}
H_{\text{quad},\text{Earth}}&=\qty{10}{\kilo\meter}+\qty{5}{\kilo\meter}\sqrt{86/7}\tan\left(\frac{\sqrt{301}k_{B}}{1000\sqrt{2}mg}-\arctan\sqrt{\frac{14}{43}}\right)\\
&\approx\qty{7179.62}{\meter}.
  \end{split}
\end{equation}
For $T_{0}=\qty{290}{\kelvin}$, $T_{1}=\qty{-1.5e-2}{\kelvin\per\meter}$,
$T_{2}=\qty{7.5e-7}{\kelvin\per\meter\squared}$,
and then 
\begin{equation}
\begin{split}
H_{\text{quad},\text{Earth}}'&= \qty{2}{\kilo\meter}\sqrt{\frac{5}{129}}(\sqrt{645}-43\tan\left(\arctan\sqrt{15/43}-\frac{188727}{20000\sqrt{645}}\right)\\
&\approx\qty{7233.91}{\meter}
\end{split}
\end{equation}
As a quick sanity check, we see that this is the same order-of-magnitude
as the constant temperature scenario.
\end{node} % Quadratic temperature

\begin{node}[Quartic temperature]\label{fluids:euler-flow-000F}%
If we tried doing the same thing with quartic temperature, 
\begin{equation*}
T(z) = T_{0} - T_{1}z + \frac{1}{2!}T_{2}z^{2} - \frac{1}{3!}T_{3}z^{3}
+ \frac{1}{4!}T_{4}z^{4},
\end{equation*}
then we would need to solve $T(\lambda_{j})=0$ for $j=1,2,3,4$. The
general solution would look like
\begin{equation}
p_{\text{quart}}(z) = C_{1}\exp\left(\frac{-1}{h}\sum_{j=1}^{4}\frac{\ln(z-\lambda_{j})}{T'(\lambda_{j})}\right).
\end{equation}
The most general form of this is not terribly enlightening, and requires
quite a bit of space to write. If we match the 1962 US Standard
Atmosphere estimates with $T(\qty{0}{\kilo\meter})=\qty{290}{\kelvin}$,
$T(\qty{15}{\kilo\meter})=\qty{213}{\kelvin}$,
$T(\qty{50}{\kilo\meter})=\qty{270}{\kelvin}$,
$T(\qty{85}{\kilo\meter})=\qty{190}{\kelvin}$,
and $T(\qty{100}{\kilo\meter})=\qty{201}{\kelvin}$, then we'd have
\begin{equation}
\begin{split}
T(z)&=290-\frac{2787509}{249900}z+\frac{6352349}{12495000} z^{2}-\frac{47947}{6247500} z^{3} +\frac{11299}{312375000} z^{4}\\
&\approx 290 - 11.1545 z + 0.508391 z^{2} - 0.00767459 z^{3} + 0.0000361713 z^{4}
\end{split}
\end{equation}
By comparison, the quadratic approximation had (converting units to make
the comparison simple)
$T_{\text{quad}}(z)\approx 290 - 15 z + 0.75z^{2}$, so we're at least
consistent with what we had with the quadratic temperature. The
approximate roots for the quartic situation is
$\lambda_{1\pm}\approx 7.52101\pm26.8494\I$ and
$\lambda_{2\pm}\approx98.5658\pm24.4373\I$. There is a great deal of
complication here, but numerically grinding through these difficulties
we find the scale atmospheric height to be
$H\approx\qty{7.48704}{\kilo\meter}$ which is on the same
order-of-magnitude as the quadratic case.
\end{node} % Quartic temperature
\end{node} % Atmosphere

\begin{node}[Steady flow]\label{fluids:euler-flow-000D}%
The notion of a steady~\zref{pde-000O} flow is specifically when
$\partial_{t}\vec{u}=\vec{0}$ in the Eulerian description.

\begin{theorem}[Bernoulli]\label{fluids:euler-flow-000E}%
The steady flow for an ideal fluid of constant density has the Bernoulli
function $H:=\rho^{-1}p+\Phi+\frac{1}{2}\|\vec{u}\|^{2}$ be constant on
streamlines, where the body force satisfies $\vec{f}=-\rho\grad\Phi$.
\end{theorem}

\begin{proof}
We see that the Euler equations for steady flow satisfy
\begin{equation*}
\vec{u}\cdot\grad\vec{u}+\rho^{-1}\grad p+\grad\Phi=0.
\end{equation*}
We use the identity from vector calculus,
\begin{equation*}
\grad(\vec{A}\cdot\vec{B})=(\vec{A}\cdot\grad)\vec{B}+(\vec{B}\cdot\grad)\vec{A}+\vec{A}\times(\grad\times\vec{B})+\vec{B}\times(\grad\times\vec{A}),
\end{equation*}
applied to $\vec{A}=\vec{B}=\vec{u}$ gives us
\begin{equation*}
\grad(\vec{u}\cdot\vec{u})=2(\vec{u}\cdot\grad)\vec{u}+2\vec{u}\times(\grad\times\vec{u}),
\end{equation*}
so
\begin{equation*}
\vec{u}\cdot\grad\vec{u}=\grad\left(\frac{1}{2}\vec\|\vec{u}\|^{2}\right)-\vec{u}\times(\grad\times\vec{u}).
\end{equation*}
Then the Euler flow becomes
\begin{equation}
\grad(\rho^{-1}p+\Phi+\frac{1}{2}\|\vec{u}\|^{2})=\vec{u}\times(\grad\times\vec{u}).
\end{equation}
Taking the dot product with $\vec{u}$ on both sides yields
\begin{equation}
\vec{u}\cdot\grad(\rho^{-1}p+\Phi+\frac{1}{2}\|\vec{u}\|^{2})=0.
\end{equation}
Consequently
\begin{equation}
\frac{\materialD}{\materialD t}(\rho^{-1}p+\Phi+\frac{1}{2}\|\vec{u}\|^{2})
=\vec{u}\cdot\grad(\rho^{-1}p+\Phi+\frac{1}{2}\|\vec{u}\|^{2})
=0.
\end{equation}
Hence the result.
\end{proof}
\end{node} % Steady flow

\begin{node}[References]\label{fluids:euler-flow-000A}%
See Landau and Lifshitz~\cite[\S3]{landau1987fluids}, Chapter 2 of
Childress~\cite{childress2009introduction}. 
\end{node} % References
\end{node}

\begin{node}[Thermodynamics of ideal fluids]\label{fluids:euler-flow-000M}%
Landau and Lifshitz~\cite[\S3]{landau1987fluids} begins by observing how
idealized fluids enjoy simplifications thanks to thermodynamics. This is
a summary of their results.

\begin{node}[Adiabatic motion]\label{fluids:euler-flow-000G}%
In the absence of heat exchange between fluid parcels (and between the
fluid and its environment) implies motion is adiabatic throughout the
fluid.  As a consequence, the motion of an ideal fluid must necessarily
be adiabatic.

\begin{corollary}\label{fluids:euler-flow-000H}%
The entropy of any fluid parcel remains constant as that parcel moves
around in space. If $s$ denotes the entropy per unit mass, the adiabatic
condition is then:
\begin{equation*}
\frac{\materialD s}{\materialD t}=\partial_{t}s+(\vec{u}\cdot\grad)s=0.
\end{equation*}
\end{corollary} % conservation of entropy density

\begin{definition}\label{fluids:euler-flow-000I}%
The quantity $\rho s\vec{u}$ is sometimes called the
\define{Entropy Flux Density} since we have the ``continuity equation''
for entropy written in terms of it:
\begin{equation*}
\partial_{t}(\rho s)+\divergence(\rho s\vec{u})=0.
\end{equation*}
When there is non-adiabatic motion, the right-hand side of this equation
is nonzero.
\end{definition}
\end{node} % adiabatic motion

\begin{node}[Isentropic fluids]\label{fluids:euler-flow-000J}%
Landau and Lifshitz observe the standard situation for fluids is that
the entropy is constant throughout the volume at some initial
instant. If this is the case, then the entropy everywhere has the same
constant value at all times and for any subsequent motion of the
fluid. In this case, the adiabatic condition may be written as:
\begin{equation*}
s=\mbox{constant}.
\end{equation*}
In this case, such motion is said to be \define{Isentropic}. When will
this occur? NASA\footnote{\url{https://www.grc.nasa.gov/www/BGH/isentrop.html}} points out this occurs when the change
in flow variables is small and gradual (e.g., the ideal fluid).
\end{node} % isentropic motion

\begin{theorem}\label{fluids:euler-flow-000K}%
The Euler equations of motion for an isentropic fluid may be rewritten
as
\begin{equation*}
\partial_{t}(\curl\vec{u})=\curl(\vec{u}\times(\curl\vec{u})).
\end{equation*}
Observe this involves only velocity, not pressure or density.
\end{theorem} % Euler equations for isentropic motion
\end{node} % Thermodynamics of ideal fluids

