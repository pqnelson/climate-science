\chapter*{Preface}

This is an attempt to collect and collate my thoughts, notes, readings
on physics, chemistry, biology, climate science, and the related
mathematics needed to understand them. It would be nice to have a
``self-contained text'' which starts with elementary algebra, builds a
computer algebra system (CAS) to formalize what has been learned,
applies the CAS to physics, and so on. But this would be a pipe-dream, a
bridge too far.\footnote{Even so, MIT 6.844 looks interesting \url{https://ocw.mit.edu/courses/6-844-computability-theory-of-and-with-scheme-spring-2003/}}

Instead I am going to write notes about climate science, introducing the
necessary fluid mechanics and partial differential equations needed to
discuss the atmosphere and ocean. I will ``lazily'' introduce notes on
vector calculus (``written as needed''), which might involve introducing
notions in calculus and vector algebra. The mathematics playing a
``supporting role'' is too vast to be swept into appendices, so I've
isolated them in their own part.

I am assuming some degree of familiarity with Lagrangian mechanics, and
the continuum limit of a linear chain of point-masses connected by
identical massless springs as the derivation of the scalar field (the
Klein--Gordon field). I will not start from ``square one'' and deduce
everything from Newton's laws of motion, I assume the reader is familiar
with basic Newtonian mechanics.

\section*{Stylistic Choices}

\subsection*{Puzzles} These are problems I don't know the answer to, but
would like to learn more about (if anyone's even thinking about them).

\subsection*{Programming} There will be numerical analysis done in these
notes. I am deliberating what language (or languages) should be
used. \FORTRAN\ has become tolerable since 1990, when it began to
imitate Pascal. But still, I am a Lisper at heart, and the heart wants
what the heart wants.

However, the use of code is to show the \emph{intent} of something, not
to actually code it up.


\subsection*{Calculations} I am going to use Dijkstra's ``structured
derivations'' for performing calculations. It is a beautiful thing. The
basic structure looks like
\begin{calculation}
  A
  \step{hint why $A=B$}
  B
  \step{hint why $B=C$}
  C
\end{calculation}
We can change the relational operator, for example
\begin{calculation}
  A
  \step[\leq]{hint why $A\leq B$}
  B
  \step{hint why $B=C$}
  C
\end{calculation}
Or we may work with sequences of equations
\begin{calculation}
  A=B
  \step[\equiv]{hint why $B=B'$, $A=A'$}
  A'=B'
  \step[\equiv]{hint why $B'=C$}
  A'=C.
\end{calculation}
It's a wonderful thing.