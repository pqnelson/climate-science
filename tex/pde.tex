% 26
\chapter{Partial Differential Equations}

\begin{node}\label{pde-0000}%
Partial differential equations usually begins by studying first-order
linear partial differential equations, then moving on to the ``big
three'' of second-order linear partial differential equations (Laplace's
equation, the wave equation, and the heat equation). Nonlinear partial
differential equations are seldom discussed in undergraduate courses
because they are so difficult that each nonlinear PDE merits its own
field of study.
\end{node}

\begin{definition}\label{pde-0001}%
We define a \define{Domain} in $\RR^{n}$ to be a non-empty, open,
connected subset $U\subset\RR^{n}$.

More generally, we can speak of domains in topological space.
\end{definition}

\begin{definition}\label{pde-000D}%
A \define{Partial Differential Equation} consists of
\begin{enumerate}
\item a domain $\Omega\subset\RR^{n}$ containing a subdomain
  $\Omega_{0}\subset\Omega$ (for the initial data);
\item a degree $k$ differential operator $P\colon C^{k}(\Omega)\to C^{0}(\Omega)$;
\item some initial data $u_{0},u_{1},\dots,u_{k-1}\in C^{j}(\Omega_{0})$ for some suitable
  $j\in\NN_{0}\cup\{\infty,\omega\}$.
\end{enumerate}
A \define{classical solution} to such a PDE consists of a function $u\in C^{k}(\Omega)$
such that $P(u)=0$ and $u|_{\Omega_{0}}=u_{0}$, $\partial_{t}u|_{\Omega_{0}}=u_{1}$,
\dots, $\partial_{t}^{k-1}u|_{\Omega_{0}}=u_{k-1}$.

\begin{node}[Lack of definition]\label{pde-000F}%
Note that most people informally refer to just the equation $P(u)=0$ as the
partial differential equation, but this is not quite correct. We need to
specify the domain, we need to specify the initial data or boundary
conditions. Only then will we have a partial differential equation.
Otherwise, there is no unique solution, and usually we end up with a
large space of solutions.

In fact, the jet comonad over $\Omega$ and the category of partial
differential equations with variables in $\Omega$ are equivalent
categories, as Marvan~\cite{marvan1986note} proved. But this abstract
nonsense forced us to realize we needed to include the domain $\Omega$
in the definition of a partial differential equation. For a review of
this treatment of PDEs, see Pugliese, Sparano, and
Vitagliano~\cite{pugliese2023vinogradovs}. 
\end{node}

\begin{node}[Boundary conditions]\label{pde-000E}%
It's not uncommon to work with boundary conditions instead of initial
data, when $\Omega$ has a boundary.
Some common ones are:
\begin{enumerate}
\item Dirichlet condition: $u$ is specified on $\bdry\Omega$;
\item Neumann condition: the normal derivative $\partial u/\partial n$ 
  is specified on $\bdry\Omega$;
\item Robin condition: $au+\partial u/\partial n$ is specified on $\bdry\Omega$.
\end{enumerate}
\end{node}
\end{definition}

\begin{node}\label{pde-000C}%
The questions that we want to answer about a partial differential
equation are generally:
\begin{enumerate}
\item Existence: Does a solution exist?
\item Uniqueness: Is the solution unique?
\item Stability: Does the solution change continuously as we change the initial data?
\end{enumerate}
When a partial differential equation has a unique solution which varies
continuously with respect to changes in the initial data, we say the
partial differential equation is \define{Well-posed}.

But there's also the important problem of \emph{how do we even find solutions?}
\begin{node}\label{pde-000B}%
Morally, there is some intuition that we ``build solutions'' out of
``wave equations''. There is some degree of truth to this, but it gets
us only so far in life.
\end{node}
\end{node}

\begin{node}[Differential Operators]\label{pde-000G}%
\begin{definition}\label{pde-000K}%
Let $\mathcal{F}$ be a function space (usually $C^{k}(\Omega)$ for some
domain $\Omega$). Let $m\in\NN$.
We define a \define{Differential Operator of order $m$}
$P=\sum_{|\alpha|\leq m}a_{\alpha}(x)D^{\alpha}$ where
$\alpha=(\alpha_{1},\dots,\alpha_{m})$ is a multi-index of non-negative
integers $|\alpha|=\alpha_{1}+\dots+\alpha_{m}$, and for each $\alpha$
the coefficients $a_{\alpha}(x)$ is a function on $\Omega$. We interpret
$D^{\alpha}$ as
\begin{equation*}
D^{\alpha} = \frac{\partial^{|\alpha|}}{\partial x_{1}^{\alpha_{1}}\cdots\partial x_{m}^{\alpha_{m}}}
\end{equation*}
\end{definition}

\begin{definition}\label{pde-000L}%
Let $P$ be a differential operator of degree $m$ on $\mathcal{F}$ over
domain $\Omega$. We call $P$ \define{Linear} if for all functions
$f_{1},f_{2}\in\mathcal{F}$ and all scalar constants [numbers]
$c_{1},c_{2}\in\RR$ (or $\CC$), we have
\begin{equation*}
P(c_{1}f_{1}+c_{2}f_{2})=c_{1}P(f_{1})+c_{2}P(f_{2}).
\end{equation*}

\begin{example}\label{pde-000H}%
We can see that $x^{n}\partial_{x}$ is a linear differential operator,
since $x^{n}\partial_{x}(c_{1}f_{1}+c_{2}f_{2})=c_{1}x^{n}\partial_{x}f_{1}+ c_{2}x^{n}\partial_{x}f_{2}$.
\end{example}

\begin{node}[Non-example]\label{pde-000I}%
Consider $P(f)=\partial_{x}(f(x)^{2})$. Then $P$ is not a linear
differential operator:
\begin{calculation}
  P(c_{1}f_{1}+c_{2}f_{2})
\step{unfold definition of $P$}
  \partial_{x}((c_{1}f_{1}+c_{2}f_{2})^{2})
\step{expanding square}
  \partial_{x}(c_{1}^{2}f_{1}^{2}+2c_{1}c_{2}f_{1}f_{2}+c_{2}^{2}f_{2}^{2})
\step{linearity of derivative}
  \partial_{x}(c_{1}^{2}f_{1}^{2})+\partial_{x}(2c_{1}c_{2}f_{1}f_{2})+\partial_{x}(c_{2}^{2}f_{2}^{2})
\step{folding definition of $P$}
  P(c_{1}f_{1}) + \partial_{x}(2c_{1}c_{2}f_{1}f_{2}) + P(c_{2}f_{2})
\end{calculation}
and since $\partial_{x}(2c_{1}c_{2}f_{1}f_{2})\neq0$ in general, we
cannot conclude that $P$ is linear.
\end{node}
\end{definition}

\begin{theorem}[Superposition principle]\label{pde-000J}%
Let $P$ be a linear differential operator of degree $m$ on $\mathcal{F}$
over $\Omega$.
\begin{enumerate}
\item If $f_{1},f_{2}\in\mathcal{F}$ satisfy $P(f_{1})=0$ and
$P(f_{2})=g$, then for any $c_{1},c_{2}\in\RR$ we have
$P(c_{1}f_{1}+c_{2}f_{2})=g$.
\item More generally, if we have $n\in\NN$
functions $f_{1}$, \dots, $f_{n}\in\mathcal{F}$ satisfying $P(f_{j})=0$,
then their linear combination $P(c_{1}f_{1}+\cdots+c_{n}f_{n})=0$.
\item If we have $P(f_{j})=g_{j}$ for $j=1,\dots,n$, then
$P(c_{1}f_{1}+\cdots+c_{n}f_{n})=g$ where $g=c_{1}g_{1}+\cdots+c_{n}g_{n}$.
\end{enumerate}
\end{theorem}
\end{node}

\begin{node}[Some terminology]\label{pde-000M}%
\begin{definition}\label{pde-000N}%
Let $\Omega$ be a domain which includes a time dimension.
Let $P$ be a differential operator of spatial variables over $\Omega$.
Let $n\in\NN$ be a constant positive integer (usually $n=1$ or $n=2$).
A partial differential equation of the form
\begin{equation*}
\partial_{t}^{n}u=P(u)
\end{equation*}
is called a \define{Time-evolution problem}.
\end{definition}

\begin{definition}\label{pde-000O}%
Let $\partial_{t}^{n}u=P(u)$ be a time-evolution problem. The
\define{Steady-state problem} associated with it is the homogeneous
differential equation $P(u)=0$ (intuitively corresponding to the
solution with $\partial_{t}u=0$).

\begin{example}\label{pde-000P}%
Steady flows~\zref{fluids:euler-flow-000D} in fluid mechanics.
\end{example}
\end{definition}
\end{node}

\transclude{pde/first-order}
\transclude{pde/wave}
\transclude{pde/heat}
