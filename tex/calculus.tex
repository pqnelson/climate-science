% 48

\chapter{Calculus}


\begin{node}\label{calculus-0000}%
I am going to work in the Eulerian spirit, relying on symbolic
computation for the most part instead of logical rigour. Certainly
rigour has its places, but it obscures rather than enlightens when it
comes to mathematics applied to science.

Rather than lay down the axioms for natural numbers, then construct the
rationals, reals, and finally complex numbers, I am going to just assume
familiarity with these topics. When forced, I will state and prove the
necessary theorems.
\end{node}

\begin{theorem}[Binomial]\label{calculus-000B}%
For any $n\in\NN_{0}$ and $x$, $y\in\CC$, we have
\[(x+y)^{n} = \sum^{n}_{k=0}\binom{n}{k}x^{k}y^{n-k}.\]
\end{theorem}

\transclude{calculus/differential}
\transclude{calculus/integral}