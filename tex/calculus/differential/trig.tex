
\section{Trigonometry}

\begin{node}[Trigonometry]\label{calculus-000S}%
Now, we can use the exponential function \zref{calculus-000D} to
introduce two new functions $\sin(x)$ and $\cos(x)$ defined by
\[\exp(\I x)=\cos(x)+\I\sin(x),\]
where $\I^{2}=-1$.

\begin{node}[Conventions]\label{calculus-000X}%
Sadly, the convention is to write $\sin^{2}(x)=\bigl(\sin(x)\bigr)^{2}$,
and similarly for other positive integer powers of $\sin(x)$, and
likewise for $\cos(x)$. It gets confusing for negative powers, because
$\sin^{-1}(x)$ may refer to the inverse function, or it may refer to
$1/\sin(x)$. I will write $\arcsin(x)$ for the inverse function of
$\sin(x)$ (the convention: ``arc'' prefixes the inverse function for
trigonometric functions) and I will have to write
$\bigl(\sin(x)\bigr)^{-n}=1/\sin^{n}(x)$ for $n\in\NN$.
\end{node}

\end{node} % trig functions

\begin{lemma}\label{calculus-000T}
We have $\cos(x+y)=\cos(x)\cos(y)-\sin(x)\sin(y)$ and $\sin(x+y)=\cos(x)\sin(y)+\sin(x)\cos(y)$.
\end{lemma}

This follows from $\exp(\I x)\exp(\I y)=\exp(\I(x+y))$.

\begin{lemma}\label{calculus-000Y}%
We have $\sin(-x)=-\sin(x)$ and $\cos(-x)=\cos(x)$.
\end{lemma}

This follows by definition of sine and cosine.

\begin{theorem}\label{calculus-000Z}%
  We have
  \[\frac{\D}{\D x}\cos(x)=-\sin(x)\]
  and
  \[\frac{\D}{\D x}\sin(x)=\cos(x).\]
\end{theorem}
This follows from the definition of the sine and cosine, as well as the
derivative of the exponential function.

\begin{definition}\label{calculus-0010}%
We define \define{Secant} $\sec(x)=1/\cos(x)$, \emph{Cosecant}
$\csc(x)=1/\sin(x)$, \emph{Tangent} $\tan(x)=\sin(x)/\cos(x)$, and
\emph{Cotangent} $\cot(x)=\cos(x)/\sin(x)=1/\tan(x)$.
\end{definition}

\begin{theorem}\label{calculus-0011}%
  We have
  \[\frac{\D}{\D x}\tan(x)=\sec^{2}(x)\]
\end{theorem}

This follows from the quotient rule Theorem~\zref{calculus-000V} and the
derivatives of sine and cosine.

\begin{node}\label{calculus-0012}%
  We have
  \[\frac{\D}{\D x}\sec(x)=\frac{\sin(x)}{\cos^{2}(x)}.\]
\end{node}

Again, the quotient rule and the derivative of cosine gives us this result.

\begin{node}\label{calculus-0013}%
  We have
  \[\frac{\D}{\D x}\csc(x)=\frac{-\cos(x)}{\sin^{2}(x)}.\]
\end{node}

\begin{node}\label{calculus-0014}%
  We have
  \[\frac{\D}{\D x}\ln(\cos(x))=-\tan(x).\]
\end{node}

\begin{proof}
By direct calculation,
\begin{calculation}
  \frac{\D}{\D x}\ln(\cos(x))
  \step{chain rule, derivative of logarithm}
  \frac{1}{\cos(x)}\frac{\D}{\D x}\cos(x)
  \step{derivative of cosine is negative sine}
  \frac{-\sin(x)}{\cos(x)}
\end{calculation}
The result follows by definition of tangent.
\end{proof}

\begin{node}\label{calculus-0015}%
We have
\[\frac{\D}{\D x}\ln(\sin(x))=\cot(x).\]
\end{node}

\begin{proof}
By direct calculation
\begin{calculation}
  \frac{\D}{\D x}\ln(\sin(x))
  \step{chain rule, derivative of logarithm}
  \frac{1}{\sin(x)}\frac{\D}{\D x}\sin(x)
  \step{derivative of sine is cosine}
  \frac{\cos(x)}{\sin(x)}
\end{calculation}
The result follows by definition of tangent and cotangent.
\end{proof}

\begin{definition}\label{calculus-0017}%
We define the 2-argument arctangent function
\begin{equation}
\atan2(y, x) =
\begin{cases}
 \arctan\left(y/x\right) &\mbox{if } x > 0, \\
 \arctan\left(y/x\right) + \pi &\mbox{if } x < 0 \mbox{ and } y \geq 0, \\
 \arctan\left(y/x\right) - \pi &\mbox{if } x < 0 \mbox{ and } y < 0, \\
 +\pi/2 &\mbox{if } x = 0 \mbox{ and } y > 0, \\
 -\pi/2 &\mbox{if } x = 0 \mbox{ and } y < 0, \\
 \mbox{undefined} &\mbox{if } x = 0 \mbox{ and } y = 0.
\end{cases}
\end{equation}
\end{definition}
