% 7
\section{Limits and O-notation}

\begin{node}\label{calculus-0002}%
The usual way to introduce derivatives is by first introducing
limits. This obscures the notion of a derivative. What we could do
instead is to introduce infinitesimals $\varepsilon\neq0$ such that
$\varepsilon^{2}=0$. But infinitesimals confuse people, apparently.
We could also introduce ``big O'' notation to make infinitesimal
reasoning more rigorous, which is what Knuth recommends.
\end{node}

\begin{definition}\label{calculus-0018}%
Let $x_{0}\in(a,b)$ --- i.e., $a<x_{0}<b$ --- and let
$f\colon(a,b)\to\RR$. We define the \define{Limit of $f$ at $x_{0}$}
to be the real number $L\in\RR$ such that: for each $\varepsilon>0$
there exists a corresponding $\delta>0$ such that for all $x\in(a,b)$ we
have $|x-x_{0}|<\delta$ implies $|f(x)-L|<\varepsilon$. We often write
\[\lim_{x\to x_{0}}f(x)=L\]
for the limit as $x\to x_{0}$ of $f(x)$. If the limit is equal to
$f(x_{0})$, then we say $f$ is \define{Continuous at $x_{0}$}. If $f$ is
continuous at each $x\in(a,b)$, then we just say $f$ is a
\define{Continuous Function}.

The intuition is that in any neighborhood of a value $f(x_{0})$, it
contains the image of a neighborhood $(x_{0}-\delta,x_{0}+\delta)$ under
$f$. 

\begin{example}\label{calculus-0019}%
Let $f(x)=x^{n}$ for any [fixed] $n\in\NN$. Then
\[\lim_{x\to x_{0}}f(x)=x_{0}^{n}.\]
Therefore $f(x)$ is continuous at $x_{0}$ for every $x_{0}\in\RR$.
\end{example}

\begin{theorem}
Let $f$, $g\colon(a,b)\to\RR$. Assume $f$ and $g$ are continuous at
$x_{0}\in(a,b)$. Then $f(x)+g(x)$ and $f(x)\cdot g(x)$ are continuous at
$x_{0}$.
\end{theorem}
\end{definition}

\begin{definition}\label{calculus-0001}%
We say $f(x)=\bigO(g(x))$ as $x\to a$ if there exists a constant $M$
such that $|f(x)|\leq M|g(x)|$ in some punctured neighborhood of $a$,
i.e., there exists some $\delta>0$ such that $|f(x)|\leq M|g(x)|$ for all $x\in(a-\delta,x+\delta)\setminus\{a\}$.

We say $f(x)=o(g(x))$ as $x\to a$ if $\lim_{x\to a}f(x)/g(x)=0$.
\end{definition}

\begin{example}
As $x-x_{0}\to 0$, a power series $f(x)=\sum^{\infty}_{k=0}a_{k}(x-x_{0})^{k}$
may be approximated as a polynomial of degree $n$ with the rest of the
series swept under the rug of big O (or little o) terms as
\[f(x) = a_{0} + a_{1}(x-x_{0}) + \dots + a_{n}(x-x_{0})^{n} + \bigO\bigl((x-x_{0})^{n+1}\bigr)\]
or
\[f(x) = a_{0} + a_{1}(x-x_{0}) + \dots + a_{n}(x-x_{0})^{n} + o\bigl((x-x_{0})^{n}\bigr).\]
\end{example}

\begin{theorem}\label{calculus-0004}%
\begin{enumerate}
\item $f(x)=\bigO(f(x))$;
\item If $f(x)=o(g(x))$, then $f(x)=\bigO(g(x))$;
\item If $f(x)=\bigO(g(x))$, then $\bigO(f(x))+\bigO(g(x))=\bigO(g(x))$;
\item If $f(x)=\bigO(g(x))$, then $o(f(x))+o(g(x))=o(g(x))$;
\item Let $c\neq0$, then $c\bigO(g(x))=\bigO(g(x))$ and $co(g(x))=o(g(x))$;
\item $\bigO(f(x))\bigO(g(x))=\bigO(f(x)g(x))$;
\item $o(f(x))\bigO(g(x))=o(f(x)g(x))$;
\item If $g(x)=o(1)$, then
  \[\frac{1}{1+o(g(x))}=1+o(g(x))\]
  and
  \[\frac{1}{1+\bigO(g(x))}=1+\bigO(g(x)).\]
\end{enumerate}
\end{theorem}

\begin{theorem}\label{calculus-0003}%
Around $0$ we have:
\begin{enumerate}
\item $x^{a}=\bigO(x^{b})$ for all $b\leq a$, and $x^{a}=o(x^{b})$ for
  all $b<a$;
\item $\bigO(x^{a})+\bigO(x^{b})=\bigO(x^{\min(a,b)})$,
  and $o(x^{a})+o(x^{b})=o(x^{\min(a,b)})$, and
\[\bigO(x^{a})+o(x^{b}) = \begin{cases}o(x^{b}) & \mbox{if }b<a\\
\bigO(x^{a}) & \mbox{if }b\geq a \end{cases}\]
\item $c\bigO(x^{a}) = \bigO(x^{a})$ and $c o(x^{a})=o(x^{a})$;
\item $x^{b}\bigO(x^{a}) = \bigO(x^{a+b})$ and
  $x^{b} o(x^{a})=o(x^{a+b})$;
\item $\bigO(x^{a})\bigO(x^{b}) = \bigO(x^{a+b})$,
  $\bigO(x^{a})o(x^{b}) = o(x^{a+b})$, and
  $o(x^{a})o(x^{b})=o(x^{a+b})$.
\end{enumerate}
\end{theorem}


\begin{node}[One-sided limits]\label{calculus:differential:limits-0001}%
\begin{definition}\label{calculus:differential:limits-0000}%
The limit as $x$ decreases in value approaching $x_{0}$ of $f(x)$ is
called the \define{Right limit} of $f$ at $x_{0}$, will be denoted by:
\begin{equation*}
\lim_{x\to x_{0}^{+}}f(x)=R
\end{equation*}
and means: for each $\varepsilon>0$ there exists a corresponding
$\delta>0$ such that for all $x\in\dom(f)$, if $0<x-x_{0}<\delta$ then
$|f(x)-R|<\varepsilon$. 
\end{definition}

\begin{definition}
The limit as $x$ increases in value approaching $x_{0}$ of $f(x)$ is
called the \define{Left limit} of $f$ at $x_{0}$, will be denoted by:
\begin{equation*}
\lim_{x\to x_{0}^{-}}f(x)=L
\end{equation*}
and means: for each $\varepsilon>0$ there exists a corresponding
$\delta>0$ such that for all $x\in\dom(f)$, if $0<x-x_{0}<\delta$ then
$|f(x)-L|<\varepsilon$. 
\end{definition}
\end{node}

\begin{node}[Limits and infinity]\label{calculus-0005}%
When we have a limit towards a value which is of the form
$1/|\varepsilon|^{n}$ as $\varepsilon\to0$, we say the limiting value
is infinity $\infty$ (if it is approaching $-1/|\varepsilon|^{n}$, its limiting
value is $-\infty$).

\begin{node}[Infinity]\label{calculus:differential:limits-0002}%
Traditionally, $\infty$ serves as more of a ``not a number'' concept
than as an infinite number. We can work with the extended real numbers
which formally adjoins $+\infty$ and $-\infty$ to $\RR$, and obeys the
expected rules.



\begin{definition}\label{calculus:differential:limits-0003}%
The \define{Extended Real Numbers} is the set $\extRR=\RR\cup\{+\infty,-\infty\}$
with the following rules of arithetic when one factor is infinite:
\begin{enumerate}
\item For any $x\in\RR$, $x\pm\infty=\pm\infty+x=\pm\infty$;
\item For any $0<x\leq+\infty$, $x\cdot(\pm\infty)=\pm\infty\cdot x=\pm\infty$;
\item For any $-\infty\leq x<0$, $x\cdot(\pm\infty)=\pm\infty\cdot x=\mp\infty$;
\item For any $x\in\RR$, $x/\pm\infty=0$;
\item For any $0<x<+\infty$, $(\pm\infty)/x=\pm\infty$;
\item For any $-\infty<x<0$, $(\pm\infty)/x=\mp\infty$.
\end{enumerate}
Note that expressions like $\infty-\infty$, $0\cdot(\pm\infty)$ and
$\pm\infty/\infty$ are still left undefined.
\end{definition}

\begin{node}\label{calculus:differential:limits-0006}%
Floating-point arithmetic attempts to emulate arithmetic with the
extended real numbers, so it's worth knowing a little bit about them.
\end{node}
\end{node} % Infinity

\begin{definition}\label{calculus:differential:limits-0004}%
Let $f\colon S\to\RR$ (where $S\subset\RR$). We define the \define{Limit
  of $f$ as $x$ approaches $\infty$} to be the value $L$ denoted
\begin{equation*}
\lim_{x\to\infty}f(x)=L
\end{equation*}
such that: for each $\varepsilon>0$ there exists a $c>0$ such that
whenever $x>c$ we have $|f(x)-L|<\varepsilon$.
\end{definition}

\begin{definition}\label{calculus:differential:limits-0005}%
Let $f\colon S\to\RR$ (where $S\subset\RR$). We define the \define{Limit
  of $f$ as $x$ approaches $-\infty$} to be the value $L$ denoted
\begin{equation*}
\lim_{x\to-\infty}f(x)=L
\end{equation*}
such that: for each $\varepsilon>0$ there exists a $c>0$ such that
whenever $x<-c$ we have $|f(x)-L|<\varepsilon$.
\end{definition}
\end{node}
