% 17
\chapter{Integral Calculus}

\section{Definition and Basic Properties}

\begin{definition}\label{calculus:integral-0000}%
Let $f$ be a continuous function. We define the \define{Antiderivative}
of $f$ to be a differentiable function $F$ such that $F'(x)=f(x)$.

Determining the antiderivative of a function is highly nontrivial, and
amounts to a sort of masochistic sudoku.

\begin{example}\label{calculus:integral-0001}%
Let $f(x)=x^{n}$ be a monomial. Then $F(x)=c+x^{n+1}/(n+1)$ is its
antiderivative, which can be verified thanks to the power rule.
\end{example}
\end{definition}

\begin{node}[Riemann integration]\label{calculus:integral-0003}%

\begin{definition}\label{calculus:integral-0004}%
Let $[a,b]=\{x\in\RR\mid a\leq x\leq b\}$ be a closed interval.
We define a \define{Partition} of $[a,b]$ to be a finite number of
points $a=x_{0}<x_{1}<\dots<x_{j}<\dots<x_{n}=b$. The \define{Norm} (or
\emph{mesh}) of the partition is $\max_{0\leq j\leq n-1}(x_{j+1}-x_{j})$
the largest length among the subintervals $[x_{j}, x_{j+1}]$ whose
endpoints are consecutive members of the partition.
\end{definition}

\begin{definition}\label{calculus:integral-0005}%
Let $[a,b]$ be a closed interval.
We define a \define{Tagged Partition} of $[a,b]$ to consist of a
partition $a=x_{0}<x_{1}<\dots<x_{n}=b$ together with numbers $t_{0}$,
\dots, $t_{n-1}$ such that $t_{j}\in[x_{j},x_{j+1}]$ for each $j=1,\dots,n-1$.
\end{definition}

\begin{definition}\label{calculus:integral-0002}%
Let $f\colon[a,b]\to\RR$ be a continuous function. Let
$P=(x, t)$ be a tagged partition of $[a,b]$. We define 
\define{Riemann Sum} of $f$ on $P$ to be the real number equal to
\[\sum^{n-1}_{j=0}f(t_{j})(x_{j+1}-x_{j}).\]
We can define the \define{Lower Riemann Sum} of $f$ on the partition
$P=x_{0}<\dots<x_{n}$ to be
\[L(f,P) =\sum^{n-1}_{j=0}\inf_{t\in[x_{j},x_{j+1}]}f(t)(x_{j+1}-x_{j}),\]
and similarly the \define{Upper Riemann Sum} of $f$ on $P$ to be
\[U(f,P) =\sum^{n-1}_{j=0}\sup_{t\in[x_{j},x_{j+1}]}f(t)(x_{j+1}-x_{j}).\]
\end{definition}

\begin{definition}\label{calculus:integral-0006}%
Let $f\colon[a,b]\to\RR$ be continuous. We define the \define{Riemann Integral}
of $f$ to be the real number $s\in\RR$ such that for each
$\varepsilon>0$ there exists a corresponding $\delta>0$ such that for
any tagged partition $x_{0}<x_{1}<\dots<x_{n}$ with tags $t_{0}$, \dots, $t_{n-1}$
whose mesh is less than $\delta$, we have
\[\left|\left(\sum^{n-1}_{j=0}f(t_{j})(x_{j+1}-x_{j})\right)-s\right|<\varepsilon.\]
When $s$ exists, we use the notation $\int^{b}_{a}f(x)\,\D x$ instead of $s$.
\end{definition}

\begin{definition}\label{calculus:integral-000B}%
If $f\colon(a,b)\to\RR$ is continuous and when restricted to the
subinterval $[a_{1},b_{1}]\subseteq(a,b)$ its Riemann integral exists,
then we say $f$ is \define{(Riemann) integrable} on $[a_{1},b_{1}]$.
\end{definition}

\begin{node}[On continuity]\label{calculus:integral-0007}%
We actually can weaken all these definitions from demanding $f$ be
continuous to merely ``piecewise continuous'', i.e., allowing countably
many discontinuities in $f$. The definitions generalize by working on
the subdomains where $f$ is continuous, then adding up all the results.
\end{node}
\end{node}

\begin{node}[Fundamental theorems of calculus]\label{calculus:integral-0008}%

\begin{theorem}\label{calculus:integral-0009}%
Let $f\colon[a,b]\to\RR$ be continuous. Define
\[F(x)=\int^{x}_{a}f(t)\,\D t.\]
Then $F$ is continuous on $[a,b]$ and differentiable on $(a,b)$ and
for each $x\in(a,b)$ we have $F'(x)=f(x)$.
\end{theorem}

\begin{theorem}\label{calculus:integral-000A}%
Let $f\colon[a,b]\to\RR$ be a function, let $F$ be its antiderivative
defined on $[a,b]$ and satisfy $f(x)=F'(x)$. If $f$ is integrable on
$[a,b]$, then
\[\int^{b}_{a}f(x)\,\D x=F(b)-F(a).\]
\end{theorem}
\end{node}

\begin{node}[Integration by parts]\label{calculus:integral-000C}%
\begin{node}[Indefinite version]\label{calculus:integral-000D}%
Let $u,v\colon\RR\to\RR$ be differentiable functions. Then
\[\int u(x)v'(x)\,\D x=u(x)v(x)-\int u'(x)v(x)\,\D x.\]
\begin{proof}
By direct calculation
\begin{calculation}
\int\frac{\D}{\D x}(u(x)v(x))\,\D x = u(x)v(x)
\step[\equiv]{apply product rule to integrand}
\int(u(x)v'(x)+u'(x)v(x))\,\D x = u(x)v(x)
\step[\equiv]{subtracting $\int u'(x)v(x)\,\D x$ from both sides}
\int u(x)v'(x)\,\D x = u(x)v(x) - \int u'(x)v(x)\,\D x.
\end{calculation}
Hence the result.
\end{proof}
\end{node}

\begin{node}[Definite version]\label{calculus:integral-000E}%
Let $u,v\colon[a,b]\to\RR$ be differentiable functions. Then
\[\int^{b}_{a} u(x)v'(x)\,\D x=u(b)v(b)-u(a)v(a)-\int^{b}_{a} u'(x)v(x)\,\D x.\]
\begin{proof}
Apply the fundamental theorem of calculus to the indefinite version, and
we get the result.
\end{proof}
\end{node}

\begin{example}\label{calculus:integral-000F}%
Consider the indefinite integral of $\ln(x)/x^{2}$. We see
\begin{calculation}
\int\frac{\ln(x)}{x^{2}}\,\D x
\step{let $u=\ln(x)$, $v=-1/x$}
-\frac{\ln(x)}{x}-\int\left(\frac{\D\ln(x)}{x}\right)\left(\frac{-1}{x}\right)\D x
\step{by $\D\ln(x)/\D x=1/x$}
-\frac{\ln(x)}{x}+\int\left(\frac{1}{x^{2}}\right)\D x
\step{performing the integral}
-\frac{\ln(x)}{x}-\frac{1}{x}.
\end{calculation}
Hence the result.
\end{example}

\end{node} % integration by parts


\begin{node}[Leibniz rule]\label{calculus:integral-000G}%
Differentiating under the integral sign means
\begin{equation*}
\frac{\D}{\D x}\left(\int^{b(x)}_{a(x)}f(x,t)\,\D t\right)
=f(x,b(x))b'(x) - f(x,a(x))a'(x)+\int^{b(x)}_{a(x)}\partial_{x}f(x,t)\,\D t.
\end{equation*}
\begin{proof}
We see that for $|h|\ll1$,
\begin{multline}
\int^{b(x+h)}_{a(x+h)}f(x+h,t)\,\D t\\
=\int^{b(x+h)}_{b(x)}f(x+h,t)\,\D t
+\int^{b(x)}_{a(x)}f(x+h,t)\,\D t
+\int^{a(x)}_{a(x+h)}f(x+h,t)\,\D t
\end{multline}
so subtracting from both sides the original integral,
\begin{multline}
\int^{b(x+h)}_{a(x+h)}f(x+h,t)\,\D t-\int^{b(x)}_{a(x)}f(x,t)\,\D t\\
=\int^{b(x+h)}_{b(x)}f(x+h,t)\,\D t
+\int^{b(x)}_{a(x)}\bigl(f(x+h,t)-f(x,t)\bigr)\D t
+\int^{a(x)}_{a(x+h)}f(x+h,t)\,\D t
\end{multline}
and dividing through by $h$
\begin{multline}
\frac{\int^{b(x+h)}_{a(x+h)}f(x+h,t)\,\D t-\int^{b(x)}_{a(x)}f(x,t)\,\D t}{h}\\
=\int^{b(x+h)}_{b(x)}\frac{f(x+h,t)}{h}\,\D t
+\int^{b(x)}_{a(x)}\frac{f(x+h,t)-f(x,t)}{h}\D t
+\int^{a(x)}_{a(x+h)}\frac{f(x+h,t)}{h}\D t
\end{multline}
then taking the limit $h\to 0$, which allows us to approximate the
integrals $\int^{b(x+h)}_{b(x)}f(x+h,t)\,\D t\approx f(x,b(x))(b(x+h)-b(x))$
(and equality on the nose when taking $h\to0$ limit thanks to the mean
value theorem for integrals)
and similarly for the integral from $a(x)$ to $a(x+h)$,
\begin{multline}
\lim_{h\to0}\frac{\int^{b(x+h)}_{a(x+h)}f(x+h,t)\,\D t-\int^{b(x)}_{a(x)}f(x,t)\,\D t}{h}\\
=\lim_{h\to0}\left(f(x+h,b(x))\frac{b(x+h)-b(x)}{h}\right.\\
+\int^{b(x)}_{a(x)}\frac{f(x+h,t)-f(x,t)}{h}\D t\\
\left.-f(x+h,a(x))\frac{a(x+h)-a(x)}{h}\right)
\end{multline}
and the result immediately follows.
\end{proof}
\end{node}
